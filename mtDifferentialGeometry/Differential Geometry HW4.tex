\documentclass{article}

\usepackage{amsmath}
\usepackage{amssymb}
\usepackage{amsthm}
\usepackage{amsfonts}
\usepackage{pgfplots}
\usepackage[a4paper,margin=1in]{geometry}

\title{Differential Geometry - Assignment 4}
\author{Ciarán Ó hAoláín - 17309103 - ciaran.ohaolain.2018@mumail.ie}

\let\nset\varnothing
\let\ddd\cdots

\newcommand{\sth}{\ \mathrm{s.th\ }}
\renewcommand{\d}{\mathrm{d}}
\newcommand{\N}{\mathbb{N}}
\newcommand{\R}{\mathbb{R}}
\newcommand{\C}{\mathbb{C}}
\newcommand{\dv}[2]{\frac{\d #1}{\d #2}}
\newcommand{\pdv}[2]{\frac{\partial #1}{\partial #2}}
\newcommand{\fdv}[3]{\left.\dv{#1}{#2}\right|_{#3}}
\newcommand{\crit}{\mathrm{Crit}\ }
\newcommand{\im}{\mathrm{Im}\ }	
\newcommand{\abs}[1]{\left|#1\right|}
\newcommand{\osub}{\subset_\mathrm{open}}

\newtheorem{theorem}{Theorem}[section]
\newtheorem{corollary}{Corollary}[theorem]
\newtheorem{lemma}[theorem]{Lemma}
\theoremstyle{definition}
\newtheorem{definition}{Definition}[section]
\theoremstyle{remark}
\theoremstyle{example}
\newtheorem*{remark}{Remark}
\newtheorem*{example}{Example}



\begin{document}
	\maketitle
	
	\begin{enumerate}
		\item \textbf{Round sphere of radius} $\mathbf{r}$.\\
		Shape operator: We have that \begin{align*}
			U(\vec{x})&=\frac{1}{r}(x_1,x_2,x_3)\\
			 S_p(\vec{v})&=-\nabla_{\vec{v}}U\\
			 &=-\frac{1}{r}\nabla_{\vec{v}}(\vec{x})\\
			 &= \frac{1}{r}\sum_{i=1}^{3}(\vec{v}[x_i])e_i\\
			 &=-\frac{\vec{v}}{r}
		\end{align*}
		Gaussian curvature:\\
		\[K=-\frac1r^2=\frac1{r^2}\]\\
		Mean curvature:\\
		\[H=\frac12(-\frac1r-\frac1r)=-\frac{1}{r}\]
		
		\pagebreak
		
		\textbf{Round cylinder of radius} $\mathbf{r}$.\\
		Shape operator: We Have that \begin{align*}
			M \in \R^3, & \qquad x_2^2+x_3^2=r^2\\
			U(\vec{x})&=\left(0,\frac{x_2}{2},\frac{x_3}{r}\right)\\
			p&=(0,0,r)
		\end{align*}
		(Any $\vec{v}\in T_pM$ has form $v_1\vec{e_1}+v_2\vec{e_2})$\\
		Set $S_{\vec{e_1}}=0,$since $U \equiv (0,0,1)$ here.\\
		\begin{align*}
			S_{\vec{e_2}}(U)&=-\frac{1}{r}(\vec{e_2})\\
			\therefore S_{\vec{v}}(U)&=v_1 S_{\vec{e_1}}(U)+v_2 S_{\vec{e_2}}(U)\\
			S_{\vec{v}}(U)&=(0,\frac{-v_2}{r},0)
		\end{align*}
		Gaussian curvature:\\
		\begin{align*}
			S(\vec{v})&=-\frac{1}{r}v_2\\
			S&=\begin{bmatrix}
			0 & 0 \\
			0 & -\frac{1}{r}
			\end{bmatrix}\\
			k_{\min} &= -\frac{1}{r}, \quad k_{\max} = 0\\
			K&=0
		\end{align*}
		Mean curvature:\\
		\[ H=-\frac1{2r} \]
		
		\pagebreak
		
		\textbf{Any flat plane in $\mathbf{\R^3}$}\\
		Shape operator:
		\begin{align*}
			M&:ax_1+bx_2+cx_3=d\\
			U&:=(a,b,c)\ \mathrm{constant}\\
			\forall p \in M,&\ S(\vec{v})\\
			&=-\nabla_{\vec{v}}U=(0,0,0)	
		\end{align*}
		Gaussian curvature:
		\[K=0\]
		Mean curvature:
		\[H=0\]
	\end{enumerate}
\end{document}