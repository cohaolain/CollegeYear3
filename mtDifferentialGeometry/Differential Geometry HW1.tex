\documentclass{article}

\usepackage{amsmath}
\usepackage{amssymb}
\usepackage{amsthm}
\usepackage{amsfonts}
\usepackage[a4paper,margin=0.75in]{geometry}

\title{Differential Geometry - Homework 1}
\author{Ciarán Ó hAoláín - 17309103 - ciaran.ohaolain.2018@mumail.ie}

\let\nset\varnothing
\let\ddd\cdots

\newcommand{\sth}{\ \mathrm{s.th\ }}
\renewcommand{\d}{\mathrm{d}}
\newcommand{\N}{\mathbb{N}}
\newcommand{\R}{\mathbb{R}}
\newcommand{\C}{\mathbb{C}}
\newcommand{\dv}[2]{\frac{\d #1}{\d #2}}
\newcommand{\pdv}[2]{\frac{\partial #1}{\partial #2}}
\newcommand{\fdv}[3]{\left.\dv{#1}{#2}\right|_{#3}}
\newcommand{\crit}{\mathrm{Crit}\ }
\newcommand{\im}{\mathrm{Im}\ }

\newtheorem{theorem}{Theorem}[section]
\newtheorem{corollary}{Corollary}[theorem]
\newtheorem{lemma}[theorem]{Lemma}
\theoremstyle{definition}
\newtheorem{definition}{Definition}[section]
\theoremstyle{remark}
\theoremstyle{example}
\newtheorem*{remark}{Remark}
\newtheorem*{example}{Example}


\begin{document}
	\maketitle
	\begin{enumerate}
		\item \[\alpha'(t)=(-4 \sin t \cos t, 2 \cos 2t , 2 \cos t)\]
		which at $t=\frac{\pi}{4}$ evaluates to \[ \left(-4 \sin \frac{\pi}{4} \cos \frac{\pi}{4}, 2 \cos \frac{\pi}{2} , 2 \cos \frac{\pi}{4}\right) = (-2,0,\sqrt{2}) \]
		\item Given $\dot{\alpha}(t)=(t^2,t,e^t)$ we obtain its integral: \[ \int (t^2,t,e^t) =  \left(\frac{t^3}{3}+c_1, \frac{t^2}{2}+c_2, e^t + c_3\right)\] and solve the following to find $\alpha$:
		\begin{align*}
			\alpha_1(0)& =c_1=1\\
			\alpha_2(0)& =c_2=0\\
			\alpha_3(0)& =1+c_3=-5 & \implies c_3=-6
		\end{align*}
		Which gives \[\alpha(t)=\left(\frac{t^3}{3}+1,\frac{t^2}{2},e^t-6 \right)\]
		as required.
		\item \begin{enumerate}
			\item A number
			\begin{align*}
				X(p)[f-3g]&=X(p)[f]+X(p)[-3g] & \left(v_p[f]:=\fdv{}{t}{t=0}f(p+t\vec{v}) \right)\\\\
				p+t\vec{v}&=(0,1,2)+t(0,1,2)\\
				&=(0,t+1,2t+2)\\
				f(p+t\vec{v})&=(2t+2)^2\\
				&=4t^2+8t+4\\
				X(p)[f] &= \fdv{}{t}{t=0}f(4t^2+8t+4)\\
				& = 8t+8\\
				& = 8\\\\
				-3g(p+t\vec{v})&=0\\
				X(p)[-3g]&=\fdv{}{t}{t=0} 0 \\
				&= 0\\\\
				X(p)[f-3g]&=X(p)[f]+X(p)[-3g]\\
				& = 8 + 0\\
				& = 8
			\end{align*}
			\item A number
			\begin{align*}
				X(p)[fg]&=(X(p)[f])g(p)+(X(p)[g])f(p)&(\mathrm{Leibniz})\\
				p+t\vec{v}&=(-1,0,1)+t(1,0,1)\\
				&=(t-1,0,t+1)\\
				f(p+t\vec{v})&=(t+1)^2+7(t-1)^2e^0\\
				&=t^2+2t+1+7(t^2-2t+1)\\
				&=8t^2-12t+8\\
				X(p)[f]&=\fdv{}{t}{t=0}8t^2-12t+8\\
				&=16t-12\\
				&=-12\\
				g(p)&=0\\
				\\
				g(p+t\vec{v})&=(t-1)(0)(t+1)\\
				&=0\\
				&\implies X(p)[g]=0\\
				\\
				X(p)[fg]&=(X(p)[f])g(p)+(X(p)[g])f(p)\\
				&=0.0\\
				&=0				
			\end{align*}
			\item Smooth function $(X.Y \in C^\infty (\R^3) \implies X.Y$ a smooth function).
			\begin{align*}
				X.Y:\R^3 & \to \R\\
				p & \mapsto X(p).Y(p)\\
				p=(x,y,z) &\mapsto (-x,y,z).(-1,-y,x)\\
				& = x -y^2 + xz
			\end{align*}
			\item Vector field $(X \times Y \in \Gamma T \R^3)$
			\begin{align*}
				X \times Y & = (xy+yz, x^2 -z,xy+y)
			\end{align*}
			\item Smooth function $X[f] \in C^\infty (\R^3) \implies X.Y$ a smooth function).
			\begin{align*}
				X[f] & = z^2 - 7x^2e^{yz}
			\end{align*}
			\item Vector field
			\begin{align*}
				X+\nabla g & = (-x,y,z) + (yz,xz,xy)\\
				& = (yz-x,xz+y,xy+z)
			\end{align*}
		\end{enumerate}
	\end{enumerate}
	
\end{document}