\documentclass{article}

\usepackage{amsmath}
\usepackage{amssymb}
\usepackage{amsthm}
\usepackage{amsfonts}
\usepackage{pgfplots}
\usepackage[a4paper,margin=1in]{geometry}

\title{Graph Theory - Assignment 4}
\author{Ciarán Ó hAoláín - 17309103}

\let\nset\varnothing
\let\ddd\cdots

\newcommand{\sth}{\ \mathrm{s.th\ }}
\renewcommand{\d}{\mathrm{d}}
\newcommand{\N}{\mathbb{N}}
\newcommand{\R}{\mathbb{R}}
\newcommand{\C}{\mathbb{C}}
\newcommand{\dv}[2]{\frac{\d #1}{\d #2}}
\newcommand{\pdv}[2]{\frac{\partial #1}{\partial #2}}
\newcommand{\fdv}[3]{\left.\dv{#1}{#2}\right|_{#3}}
\newcommand{\crit}{\mathrm{Crit}\ }
\newcommand{\im}{\mathrm{Im}\ }	
\newcommand{\abs}[1]{\left|#1\right|}
\newcommand{\osub}{\subset_\mathrm{open}}

\newtheorem{theorem}{Theorem}[section]
\newtheorem{corollary}{Corollary}[theorem]
\newtheorem{lemma}[theorem]{Lemma}
\theoremstyle{definition}
\newtheorem{definition}{Definition}[section]
\theoremstyle{remark}
\theoremstyle{example}
\newtheorem*{remark}{Remark}
\newtheorem*{example}{Example}


\begin{document}
	\maketitle
	
	\begin{enumerate}
		\item Since $A$ and $B$ are the bipartite sets of $G$, at no point can a Hamiltonian walk of $G$ follow a vertex in $A$ or $B$ with a vertex in the same set.\\
		Therefore, we must keep going back and forth between the sets $A$ and $B$.\\
		Bipartite cycles must have even length.
		\item \begin{enumerate}
			\item \begin{align*}
				A&=\{(0,0,0,0),(1,1,0,0),(0,0,1,1),()\}
			\end{align*}
		\end{enumerate}
	\end{enumerate}
	
\end{document}