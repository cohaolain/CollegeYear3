\documentclass{article}

\usepackage{amsmath}
\usepackage{amssymb}
\usepackage{amsthm}
\usepackage[a4paper,margin=0.75in]{geometry}

\title{ST461 Probability - Assignment 2}
\author{Ciarán Ó hAoláín - 17309103 - ciaran.ohaolain.2018@mumail.ie}

\let\nset\varnothing
\let\ddd\cdots

\newtheorem{theorem}{Theorem}[section]
\newtheorem{corollary}{Corollary}[theorem]
\newtheorem{lemma}[theorem]{Lemma}
\theoremstyle{definition}
\newtheorem{definition}{Definition}[section]
\theoremstyle{remark}
\newtheorem*{remark}{Remark}
\theoremstyle{example}
\newtheorem*{example}{Example}

\renewcommand{\d}{\ \mathrm{d}}
\newcommand{\dd}{\mathrm{d}}

\makeatletter
\newcommand{\skipitems}[1]{%
	\addtocounter{\@enumctr}{#1}%
}
\makeatother

\begin{document}
	\maketitle
	\begin{enumerate}
		\skipitems{1}
		\item \begin{enumerate}
		 \item \[ 0.55^5 \cdot 0.45 =  0.022647797 \]
			 \item \[ 0.55 \cdot 0.45 \cdot {5 \choose 3} \cdot 0.45^3 \cdot 0.55^2 \cdot 0.45 = 0.030700867\]	
		\end{enumerate}
		\item $P($Machine finishes work with two engines) $= 1-P($both engines fail) \[1-(1-p)^2=-p^2+2p\]\\
		$P($Machine with 4 engines finishes work): \[ \sum_{k=2}^{4} {4 \choose k} p^k \cdot (1-p)^{4-k} = 3 p^4 - 8 p^3 + 6 p^2 = p^2 \cdot (6 - 8p + 3p^2) \]
		Solving for $0 < p < 1$,  $\quad p^2 \cdot (6 - 8p + 3p^2) > -p^2+2p$ :
		\begin{align*}
			\quad p^2 \cdot (6 - 8p + 3p^2) &> -p^2+2p\\
			3 p^4 - 8 p^3 + 6 p^2 & > -p^2 + 2p\\
			(p-1)^2(3p-2) & > 0 & ( 0 < p < 1)	\\
			3p-2 &> 0\\
			p > \frac{2}{3}
		\end{align*}
		$\therefore$ a four-engine machine is preferable to a two-engine when $p > \frac{2}{3}$.
		\skipitems{1}
		\item \begin{enumerate}
			\item Clearly $f(x)\geq0 \forall x, \alpha$.\\
			We need to also verify that $\int_{-\infty}^{\infty}f(x) \d x=1$ as follows:
			\begin{proof}
				\begin{align*}
					\int_{-\infty}^{\infty}f(x) \d x &=\int_{-\infty}^{-1}0 \d x + \int_{-1}^{1}f(x) \d x + \int_{1}^{\infty}0 \d x\\
					& = \int_{-1}^{1}f(x) \d x\\
					& = \frac{1}{2} \int_{-1}^{1} \left(1 + \alpha x\right) \d x\\
					& = \frac{1}{2} \left[ x + \frac{\alpha x^2}{2} \right]^1_{-1}\\
					& = \frac{1}{2} \left( 1 + \frac{\alpha}{2} + 1 - \frac{\alpha}{2} \right)\\
					& = \frac{1}{2} \cdot 2\\
					& = 1
				\end{align*}
			as required.
			\end{proof}
			$\therefore$ we have that $f$ is a probability density function.
			\item The CDF can be found as follows:
			\begin{align*}
				F(x)&=\int_{-\infty}^{x}f(u) \d u\\
				&=\int_{-\infty}^{-1}0 \d u + \int_{-1}^{x}f(u) \d u\\
				& = \int_{-1}^{x}f(u) \d u\\
				& = \frac{1}{2} \int_{-1}^{x} \left(1 + \alpha u\right) \d u\\
				& = \frac{1}{2} \left[ u + \frac{\alpha u^2}{2} \right]^x_{-1}\\
				& = \frac{1}{2} \left(\alpha\left(\frac{x^2 - 1}{2}\right) + x + 1\right)
			\end{align*}
			as required.
			\item At first quartile, CDF ($F(x)$)is 0.25:
			\begin{align*}
				\frac{1}{4}&=\frac{1}{2}\left(\alpha\left(\frac{x^2-1}{2} + x + 1\right)\right)\\
				\frac{1}{2 \alpha} &= \frac{x^2-1}{2} + x + 1\\
				\frac{1}{\alpha} &= x^2 + 2x + 1\\
				x &= \frac{\pm \sqrt{1 - \alpha + \alpha^2} -1}{\alpha}
			\end{align*}
			At median, $F(x)=0.5$, so as above:
			\[
				x = \frac{\pm\sqrt{\alpha^2+1} -1}{\alpha}
			\]
			And finally, at median, $F(x)=0.75$, so as above:
			\[
				x=\frac{\pm \sqrt{1 + \alpha + \alpha^2} -1 }{\alpha}
			\]
		\end{enumerate}
	\end{enumerate}
\end{document}