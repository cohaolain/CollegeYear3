\documentclass{article}

\usepackage{amsmath}
\usepackage{amssymb}
\usepackage{amsthm}
\usepackage{amsfonts}
\usepackage{pgfplots}
\usepgfplotslibrary{fillbetween}
\usepackage[a4paper,margin=0.75in]{geometry}

\title{ST461 Probability - Assignment 4}
\author{Ciarán Ó hAoláín - 17309103 - ciaran.ohaolain.2018@mumail.ie}

\let\nset\varnothing
\let\ddd\cdots

\newtheorem{theorem}{Theorem}[section]
\newtheorem{corollary}{Corollary}[theorem]
\newtheorem{lemma}[theorem]{Lemma}
\theoremstyle{definition}
\newtheorem{definition}{Definition}[section]
\theoremstyle{remark}
\newtheorem*{remark}{Remark}
\theoremstyle{example}
\newtheorem*{example}{Example}
\newcommand{\N}{\mathbb{N}}

\newcommand{\uni}{\mathrm{Uniform}}
\newcommand{\Exp}{\mathrm{Exp}}
\newcommand{\Var}[1]{\mathrm{Var}\left(#1\right)}
\newcommand{\Cov}[1]{\mathrm{Cov}\left(#1\right)}
\newcommand{\Cor}[1]{\mathrm{Cor}\left(#1\right)}
\newcommand{\Binomial}[1]{\mathrm{Binomial}\left(#1\right)}
\newcommand{\Poisson}[1]{\mathrm{Poisson}\left(#1\right)}
\newcommand{\otherwise}{\mathrm{otherwise}}

\renewcommand{\d}{\ \mathrm{d}}
\newcommand{\dd}{\mathrm{d}}

\makeatletter
\newcommand{\skipitems}[1]{%
	\addtocounter{\@enumctr}{#1}%
}
\makeatother

%\setlength{\tabcolsep}{20pt}
\renewcommand{\arraystretch}{1.5}

\begin{document}
	\maketitle
	\begin{enumerate}
		\item
		\begin{enumerate}
			\item \[\frac{2(0)+3(1)+1(2)+2(3)+1(4)+1(5)}{10}=2\]
			\item \[ \frac{3(1)+2(2)+6(3)+4(4)+5(5)}{20} = \frac{66}{20}=\frac{33}{10} = 3.3\]
		\end{enumerate}
	\skipitems{1}
		\item \begin{align*}
			f_X(x)&=\frac{3}{2} \int_{0}^{1}x^2+y^2\d y & 0 < x < 1\\
			& = \frac32\left[x^2y+\frac{y^3}{3}\right]^1_0\\
			& = \frac12\left(3x^2+1\right)\\
			\\
			f_X(x)&=\begin{cases}
				\frac12\left(3x^2+1\right) & 0 < x < 1\\
				0 & \otherwise
			\end{cases}\\
			f_Y(y)&=\begin{cases}
			\frac12\left(3y^2+1\right) & 0 < y < 1\\
			0 & \otherwise
			\end{cases} & \mathrm{by\ symmetry}\\
			\\
			E[X]&=\int_{0}^{1}x f_X(x)\\
			& = \frac12 \int_{0}^{1}3x^3+x \d x\\
			&= \frac12 \left[\frac{3x^4+2x^2}{4}\right]_0^1\\
			&=\frac{3+2}{8} = \frac58  & \implies E[Y]=\frac58\ \mathrm{by\ symmetry}\\
			\\
			E[XY]&=\int_{0}^{1}\int_{0}^{1}xyf_{XY}(x,y)\d x \d y\\
			& = \frac32 \int_{0}^{1} \int_{0}^{1} x^3y+y^3x \d x \d y\\
			&= \frac32 \int_{0}^{1} \left[\frac{x^4y}{4}+\frac{x^2y^3}{2}\right]_0^1 \d y\\
			&= \frac32 \int_{0}^{1} \frac{y}4 + \frac{y^3}{2} \d y \\
			&= \frac32 \left[\frac{y^2+y^4}{8}\right]_0^1\\
			&= \frac32 \left(\frac14\right) = \frac38\\
		\end{align*}
		\begin{align*}
			E[X^2] & = \int_{0}^{1}x^2 f_X(x) \d x\\
			&= \frac12 \int_{0}^{1}x^2 \left(3x^2+1\right) \d x & 0 < x < 1\\
			&= \frac12 \int_{0}^{1}x^4+x^2 \d x\\
			&= \frac12 \left[\frac{3x^5}{5}+\frac{x^3}{3}\right]_0^1\\
			&= \frac12 \left(\frac35+\frac13\right)\\
			&= \frac7{15}\\
			\\
			\Var{X}&=E[X^2]-\mu^2\\
			&= E[X^2]-(E[X])^2\\
			&= \frac7{15}-\left(\frac58\right)^2\\
			&= \frac7{15}-\frac{25}{64}\\
			&= \frac{73}{960} & \implies \Var{Y}=\frac{73}{960}\\
			\\
			\Cov{X,Y}&=E[XY]-\mu_X\mu_Y\\
			&= E[XY]-E[X]E[Y]\\
			&= \frac38 -\frac{25}{64}\\
			\\
			\Cor{X,Y}&=\frac{\frac{3}{8}-\frac{25}{64}}{\frac{73}{960}}\\
			&= \frac{-15}{73} & -1 \leq \frac{-15}{73} \leq 1 \quad \checkmark
		\end{align*}
		\skipitems{1}
		\item \begin{enumerate}
			\item We can see that $P(X=x)=\frac1n$.\\
			Therefore we can compute that \begin{align*}
				E[X]&=\sum_{x \in X} x \left(P(X=x)\right)\\
				&=\sum_{x \in X} x \left(\frac1n\right)\\
			\end{align*}
			We want to prove that this is the same as $\frac{n+1}{2}$. We do so by induction:
			\begin{align*}
				E[X] &= \frac{n+1}{2} = \sum_{x \in X}x\frac{1}{n}\\
				& \iff \sum_{x=1}^{n}x = \frac{n(n+1)}{2}
			\end{align*}
			First prove the n=2 case:
			\[ 1+2 = \frac{2(2+1)}{2} \iff 3 = 3 \ \checkmark\]
			Now let $k=n$. Assume true for $k$, and prove for $k+1$.
			\begin{align*}
				\sum_{x=1}^{k} + k + 1 &= \frac{k(k+1)}{2} + k_1\\
				\sum_{x=1}^{k+1} &= \frac{2k+2+k(k+1)}{2}\\
				&= k^2+3k+2\\
				&= \frac{(k+2)(k+1)}{2}
			\end{align*}
			Which gives that the statement is true for $n=k+1$. Therefore, by induction, it is true that $E[X]=\frac{n+1}{2} \forall n \in \N , n\geq 2$
				\item \[E[X/n]=\frac{n+1}{2n},\qquad \Var{X}=(\frac{n^2-1}{12n})\]
		\end{enumerate}
	\end{enumerate}
\end{document}