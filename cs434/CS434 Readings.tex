\documentclass{article}

\usepackage{amsmath}
\usepackage{amssymb}
\usepackage{amsthm}
\usepackage{pgfplots}
\usepackage{cancel}
\usepackage[a4paper,margin=1in]{geometry}
\usepackage{hyperref}

\title{CS434 Readings in the Foundations of Computer Science}
\author{Ciarán Ó hAoláín}

\let\nset\varnothing
\let\ddd\cdots

\newcommand{\sth}{\ \mathrm{s.th}\ }
\newcommand{\R}{\mathbb{R}}
\newcommand{\N}{\mathbb{N}}
\newcommand{\Z}{\mathbb{Z}}
\newcommand{\C}{\mathbb{C}}
\renewcommand{\H}{\mathbb{H}}
\newcommand{\quotient}[2]{{\raisebox{.2em}{$#1$}\left/\raisebox{-.2em}{$#2$}\right.}}
\newcommand{\abs}[1]{\left|#1\right|}
\newcommand{\cts}{cts. }
\newcommand{\ctsm}{\ \mathrm{cts.}\ }
\newcommand{\tm}{\ \mathrm{TM}\ }
\newcommand{\sforall}{\ \forall}
\newcommand{\dom}{\mathrm{dom}}
\newcommand{\tr}{\leq_\mathrm{T}}
\newcommand{\halt}{\mathrm{Halt}_\mathrm{TM}}
\newcommand{\halts}{\downarrow}
\newcommand{\ce}{\ \mathrm{c.e.}\ }
\newcommand{\tdeg}{\deg_\mathrm{T}}
\newcommand{\teq}{\equiv_\mathrm{T}}
\newcommand{\mor}{\leq_\mathrm{m}}
\newcommand{\oor}{\leq_\mathrm{1}}
\newcommand{\ran}{\mathrm{ran}}

\renewcommand{\emptyset}{\varnothing}


\newtheorem{theorem}{Theorem}[section]
\newtheorem{corollary}[theorem]{Corollary}
\newtheorem{lemma}[theorem]{Lemma}
\theoremstyle{definition}
\newtheorem{definition}[theorem]{Definition}
\theoremstyle{remark}
\newtheorem*{remark}{Remark}
\theoremstyle{example}
\newtheorem*{example}{Example}

\makeatletter
\newcommand{\skipitems}[1]{%
	\addtocounter{\@enumctr}{#1}%
}
\makeatother

\begin{document}
	\maketitle
	\section*{Lecture 1 (20.02.05)}
	This year: Only 1 paper, post's programming, then switch to modern book.
	\section{Post's Programming}
	\textbf{Post's question:}\\
	CS370: $\phi_0, \phi_1, \phi_2 \ddd$ \textit{effective list} of all TMs. A TM $\leftrightarrow \{0,1\}^\star \leftrightarrow \N$.\\
	\[\mathrm{i.e.\ }\exists \tm \phi \sth \sforall i \in \N \sforall x : \phi(i,x) \cong \phi_i(x) \]
	Computable decidable set: $A \subseteq \N \iff \exists \tm \phi \sforall x : x \in Q \implies \phi(x) = 1$\\
	c.e. $A = \dom(\phi) = \{x:\phi(x)\downarrow\}$\\
	$A \tr B \iff \exists$ oracle $\tm \phi^B$ that decides $A$.\\
	$\halt = \{ <M,w> \mid M(w) \halts \}$
	But we say: $\phi':=\{e:\phi_e(e)\halts\}$.\\
	$\phi'$ is c.e. (r.e.) but not computable (recursive). (Post).\\
	We say $\phi'$ is T-complete.\\
	Recall $\forall A \ce : A \tr \phi'$.\\
	Post Question: $\exists?$ a $\ce$, not computable, not complete set. i.e. not $\tdeg(\phi')$ or $\ce$.\\
	Not computable, not complete.\\
	$\tdeg (A) = \{B:B \teq A\} = \{B:A \tr B, B \tr A\}$.\\
	\\
	Spoiler: Post's \textit{program} will fail (proof via maximal sets).\\
	But Post's question is \textit{yes} (priority method, russian and american guy).\\
	\\
	\textbf{Post's idea:}
	Start with $A = \N$.:P 
	
	\section*{Lecture 2 (2020.02.07)}
	Notion of computability, e.g. a TM, but others exist such as recursive functions (Kleene).
	\begin{definition}
		$A$ is r.e. if $A=\mathrm{range}(f)$ where $f$ is computable.\\
		i.e. $f:\N \to \N$, total, $\exists \tm \forall n : \phi(n)=f(n)$.\\
		$A=\{f(0),f(1),\ddd\}$
	\end{definition}
	ex. $A=\N$, choose $f(n)=n$.
	One formal way of generating a set of numbers, encode numbers into unary, $n \leftrightarrow 1^n$, doing stuff?\\
	Each normal system uniquely defines a set, possibly null,
	of positive integers.\\
	$\tm M \leftrightarrow$ an r.e. set dom($M$).\\
	\textbf{Post.} normal system $\leftrightarrow$ \verb|_____|
	 Each normal system uniquely defines a set, possibly null,
	of positive integers, namely the integers represented by those derived
	strings which are strings of l's only. It can then be proved that every
	recursively enumerable set of positive integers is the set of positive
	integers defined by some normal system, and conversely.\\
	$O:B_1,B_2,B_3,\ddd\\
	\mathrm{modern:}\ \phi_1,\phi_2, \phi_3 \ddd$\\
	\\
	$\aleph_0 = \abs{\N}\\
	2^{\aleph_0}=\abs{\R}$\\
	Given $A$, the characteristic function of $A$
	\[\chi_A(n)=\begin{cases}
		0 & n \notin A\\
		1 & n \in A
	\end{cases}\]
	$A$ recursive $\iff \chi_A$ is recursive.\\
	$A \teq B,\ B \in \tdeg(A)\\
	A <_T B \iff A \tr B, B \nleq_T A$\\
	\\
	
	\begin{theorem}
		$S$ recursive $\iff$ $S, \bar{S}$ r.e.
	\end{theorem}
	\begin{proof}
		$\implies S$ is recursive.\\
		1,2,3,4,...\\
		$f(1),f(2),f(3),f(4),\ddd$\\
		$\impliedby$: given $n$
	\end{proof}

	\begin{theorem}
		$A$ is recursive $\iff A$ is r.e. in order.
	\end{theorem}
	\begin{proof}
		Stuff
	\end{proof}

	\begin{theorem}
		$\forall A$ r.e., infinite, $\exists$ recursive $B \subseteq A$
	\end{theorem}
	\begin{proof}
		if $n_1,n_2,n_3,\ddd$ enum. of $A$ (need not be ordered, can have repetitions)\\
		Now we only write down $n_2$ if greater than $n_1$, continue until find greater than $n_1$.
		Why eventually $n_1 < n_{27} < n_{254} < \ddd$ i.e. always find one satisfying because the set if inifnite.
		Is enumeration of a infinite, subset $B$ of $A$ in order i.e. $B$ is recursive.
	\end{proof}

	\section*{Lecture 3 (2020.02.12)}
	\textbf{1.7.15}
	\begin{definition}
		c.e. set $A$ is \textbf{creative} $\iff \exists$ computable 1-1 $p \sth \forall e [W_e \cap C = \emptyset \to p(e) \in \overline{W_e} \cap \bar{C} ]$
	\end{definition}
	\textbf{Idea:} $\bar{C}\qquad W_e$ that tries to $\bar{C}$, then $p(e)$ gives us an example where $W_e$ fails.\\
	i.e. $p(e) \notin W_e$ but $p(e) \in \bar{C}$\\
	N.B. $D_e := \dom (\phi_e)$\\
	\\
	\textbf{1.7.16}\\
 	$\Phi'$ is creative via $p(e):=e$.
 	\begin{proof}
 		(Note: $\forall e [W_e \cap C = \emptyset \to p(e) \in \overline{W_e} \cap \bar{C} ]$)\\
 		Let $e \in \N$\\
 		If $e \in \Phi' \to e \in W_e$
 		$\Phi'=\{e: e \in W_e\}$
 		if $e \in \Phi'$ with $W_e \cap \Phi = \phi$\\
 		$\to e \notin W_e$ i.e. $e \in \overline{W_e}$ so $p(e) \in \overline{W_e} \cap \overline{\Phi'}$.
 	\end{proof}
 
 	$A \mor B$. $f:\sigma* \to \sigma*$, $f$ is c.e.\\
 	$w \in A \iff f(w) \in B$\\
 	$A \tr B$ different because can use oracle any number of times, but with many-one reducibility this is only allowed one time.\\
 	$A \oor B$:$f$ is 1-1 computable.\\
 	\\
 	$Phi'$ is $m$-complete (actually 1-complete).\\
 	$A$ is c.e. $\iff$ $A \mor \Phi'$\\
 	\begin{proof}
 		$\impliedby$\\
 		$A(x)=\Phi'(h(x))$ for some computable $h$.\\
 		$\phi_{h(x)}(h(x)) \halts$.\\
 		$\Psi(x):=\phi_{h(x)}(h(x))$
 	\end{proof}
 	
 	\textbf{missing A LOT}
 	
 	\textbf{Lecture 2020.02.21}
 	\textbf{Use Principle:}\\
 	Convention: $\phi^Y_{e,s}(x) = y$\\
 	$e,x,y$ and all queries to $Y$ are $< s$.\\
 	\[\phi_e^Y(x) \iff \exists s \phi_{e,s}^Y(x)=y \]
 	use $\phi_{e,s}^Y(x)=1+$size largest query (at stage $s$)\\
 	use $\phi_e^Y(x)=\begin{cases}
 		1+\underline{\ \ \ \ \ }\\
 		\mathrm{undefined\ if\ } \phi_e^Y(x) \uparrow
 	\end{cases}$\\
 	If $\phi_e^Y(x) \halts$ then it will only query $Y$ on bits within $\underbrace{Y\uparrow}_{=:\phi\ \mathrm{string}}$ use $\phi_2e^Y(x)$\\
 	\[[A\uparrow \cap :=A(0) \ddd A(n-1)]\] 	
 	$\phi_e^\sigma(x)=\phi_e^{Y}\quad \abs{\sigma}=\mathrm{use}\ \phi_e^Y(x)$.\\
 	Why? Construct c.e. oracles $A$.\\
 	\[\phi_{e,s}^{A_s}(n)\halts\quad \phi_{e,s+1}^{A_{s+1}}(n)\halts \quad \ddd \quad \phi_{e}^{A}(n)\halts\]
 	
 	$f:\N \to \N$. \\
 	$F \leq_\mathrm{wbt} Y$ weak truth table reducible\\
 	if $\forall n f(n)=\phi_e^Y(n)$ and\\
 	use $\phi_e$ us bounded by a computable function.\\
 	(i.e. $\exists g$: comp. $\sth\ \forall n:$ use $\phi_e^Y(n) \leq g(n)$)\\
 	\\

 	$\leq_\mathrm{wbt} \neq \leq_\mathrm{tt}$.\\
 	We could have $f \leq_{\mathrm{wbt}}$ via $\phi_e$ i.e. $\phi_e^Y$ is total.\\
 	But $\phi_e^Z$ might be non-total, for some set $Z_1$.
 	
 	$f \leq_\mathrm{tt} A$ via $\phi$ then run time of $\phi$ is bounded by some computable function. \textbf{Need rest of this theorem}
 	\\
 	\\
 	\textbf{Presentation:}
 	$\Delta_2^0$ sets and Shoenfield Limit Lemma\\
 	\textbf{motivation:}
 	in c.e. for a set $Z$ for each input $x$, $Z_s(x)$ can change at most once.\\
 	$\Delta_2^0$ sets allow arbitrary finite number of changes.\\
 	\begin{definition}
 		We say that a set $Z$ is $\Delta_2^0$ if there is a sequence of strong indices $Z_g, g \in \N$, s.t.h $Z_s \subset [0,5)$ \& $Z(x)=\lim_s(Z_s(x))$ ($Z_s$ can be viewed as a bitset).\\
 		
 	\end{definition}
 	\textbf{Note:} $(Z_s)$is a computable approximation for $Z$.\\\\
 	\textbf{Notation:} Given an expression $E$, approximated at stages $E[S] =$ value of $R$ at tend of stage $s$. 
 	
 	e.g. Given $Z$, $\Delta_2^0$, $\phi_{e,s}^{Z_s}(x)=\phi_e^Z(x)[g]$
 	
 	\begin{definition}
 		We say an expression $E$ is stable at time $s$ if $\forall t \geq s$, $E(t)=E(s)$.
 	\end{definition}
 
 	\[Z \Delta)2^0 \iff | \tr \theta' \]
 	
 	\begin{proof}
 		$\impliedby$ Let $\phi_e^{\theta'}(x) $ be a turing function for $Z$
 		\[ Z_s = \{s < x \mid \phi_e^{\theta'} (x)[s] =1 \} \]
 	\end{proof}
	\section*{Lecture 2020.02.26}
	\textbf{Ex. 1.2.24} (which was apparently a homework or something?)\\
	$A_0, A_1$, c.e., $C=A_1 \cup A_0,\ A_1 \cap A_0 = \emptyset$\\
	Show that $C\equiv_\mathrm{wtt}$ $A_1 \oplus A_0,\ A \oplus B = \{2n \mid n \in A\} \cup \{2n+1 \mid n \in B\}$\\\\
	$\bullet C \leq_\mathrm{wtt} A_0 \oplus A_1\\
	n \in C \iff n \in A_0\ \mathrm{or}\ A_1 \implies 2n \in A_0 \oplus A_1\ \mathrm{or}\ 2n+1 \in A_0 \oplus A_1$\\
	use$\phi(n) \leq 2n+1$\\
	\\
	$\bullet A_0 \oplus A_1 \leq_{\mathrm{wtt}} C\\
	\delta(n) = \begin{cases}
		\frac{n-1}{2} & n\ \mathrm{odd}\\
		\frac{n}{2} & n\ \mathrm{even}
	\end{cases}\\
	n \in A_0 \oplus A_1 \iff \delta(n) \in C\\
	A_0 \oplus A_1 \mor C \implies \mathrm{wtt}$
	Or something simpler, find the thing.\\
	\\
	\textbf{Ex. 1.2.25}\\
	$\exists Z : f \leq_{\mathrm{tt}} Z \iff \exists$ comp. $h\ \sth : \forall n\ f(n) \leq h(n)$.\\
	$\impliedby:$ use graph.\\
	$\implies$: $\phi_e^Z(n)=f(n)$\\ last line?\\
	\\
	\begin{lemma}
		$Z is \delta_2^0 \iff Z \tr  \theta'$
	\end{lemma}
	\begin{proof}
		$\implies$ Define a set $C$ c.e. $\sth Z \tr C$. This is a sufficient $C$ is called change set.\\
		If $Z_s(x) \neq Z_{s+1}$ we put $<x,i>$ into $C$ where $i$ is the smallest number $\sth <x,i> \in C$\\
		Big diagram, very lost, missing stuff.
	\end{proof}
	1.4.3\\
	\\
	Ecercise: $Z$ is 1-c.e. iff it is ce.\\
	$Z$ is 2-c.e. iff $Z=A-B$, $A,B$ c.e.\\
	\\
	\\
	\textbf{Absolute computational complexity of sets}
	Idea: Lowness of a set $A$\\
	set is computational, is not useful as an oracle. set is if $A' \tr \Phi', A' \geq_\mathrm{T} O'$\\
	\begin{definition}
		$f,g:\N\to\R\\
		f$ domanites $y$ if $f(n) \geq g(n)$ for sufficiently large $n$.
	\end{definition}

	\begin{definition}
		Set $C$ is superlow if:\\
		$C' \equiv_\mathrm{TT} \phi'\\
		C' \leq_\mathrm{TT} \phi'\\
		C'$ is w-c.e.
	\end{definition}

	\begin{definition}
		$A$ is computably dominated if $g \tr A$ is dominated by a computable function.
	\end{definition}

	\section*{Lecture 2020.02.28}
	$p_n$ is (\textit{nth element of set})
	\begin{definition}
		Set $E \subseteq \N$ is hyperimmune if $R$ is infinite, and $p+E$ is not dominated by a computable function.\\
		$p_E$ is a listing of $E$ in order of magnitude.
		\[E=\{0,10^6,10^3,6\ddd\},\ p_E=\{0,7,10^3,10^6,\ddd\}\ p+E:\N \to \N\]
	\end{definition}
	\begin{theorem}
		$A$ not computably dominated $\iff$ $\exists$ hyperimmune $E \teq A$.
	\end{theorem}
	\begin{proof}
		$(\implies)$ Suppose $g \tr A$, $g$ not computably dominated by some computable function.\\
		Let $E=\ran(h)$, $h$ defined as follows\\
		$h(0)=0$, for $n \in \N$\\
		$h(2n+1)=h(2n)+g(n)+1 \to h$ dominates $g\\
		h(2n+2) = h(2n+1) + p_A(n) + 1$\\
		$E \teq h \teq A$\\
		$(\impliedby)$ Immediate since $p_E \tr E$ 
	\end{proof}

	\begin{theorem}[1.2.22]
		$f \leq_{\mathrm{tt}}A \iff$ there is a turing function $\Phi$ and computable function $t\ \sth f=\Phi^A$ and the no. of steps needed to compute $f$ is bounded by $t(n)$, $t$ dominates $f t(n) \geq f(n)$.
	\end{theorem}
	\begin{theorem}
		$A$ is computable dominated $\iff$ for each $f$, $f \tr A \to f \leq_{\mathrm{tt}} A$
	\end{theorem}
	\begin{proof}
		(Philippe: Have bound on use, total no matter oracle)\\
		($\implies$) Suppose $f = \Phi^A$\\
		Let $g(n)=$ (\textbf{missing end cause of wonky symbol})
	\end{proof}
	
	\begin{theorem}
		$\forall A \in \Delta_2^0 \implies A$ is not computably dominated.
	\end{theorem}
	\begin{proof}
		$A$ c.e. $g(n)=\min_s:A_{s(n)}$ is stable $\leq h(n)$\\
		$A_0(n)=0,\qquad A_{h(n)}(n)\\
		A \in \Delta_2^0 A\qquad s' > s$\\
		\textbf{missing stuff cause he fast}\\\\
		$(A_s)_s \in \N$.\\
		$g(s) \cong \mu t \geq s . \underbrace{A t \uparrow s}_{\mathrm{the\ approx\ of\ }A\mathrm{\ at\ stg\ t\ for\ element\ }<s\mathrm{\ is\ correct}} = A \uparrow s\\
		$obv: $A \geq_\mathrm{T}g\qquad \implies \exists$ comp. $f:\forall n : f(n) \geq g(n)$.\\
		obv: given $g$, $\mu$ can compute $A$: $n \in A$?\\
		$g(n+1)=\ddd$ unfortunately I only have an upper bound. $f > g$.\\
		
		$f(s) \geq g(s)$\\
		2nd idea: eventually $A_t \uparrow s$ is stable also $\exists t \in [\underbrace{n}_{\ni t, t'}, f(n)].\ A_t \uparrow n$ is correct.\\
		\\
		1 extra idea:\\
		find an input $n$ the least stg $s>n$\\
		$\sth\ A_s(n)\ \forall u[s,f(s))$		
	\end{proof}

	\subsection*{Lecture 2020.03.04}
	\textbf{Arithmetical Hierarchy:}
	1.4.10: $A$ is\\
	$\sum_n^0$ if : $x \in A \leftrightarrow \exists y_1 \forall y_2 \exists y_3 \ddd Q_{y_n}\ R(x,y_1,\ddd,y_n)$ (computable relation)\\
	$\pi_n^0$ if : $x \in A \leftrightarrow \forall y_1 \exists\ddd $ (computable relation)\\
	$\bar{Z}$:c.e. sets relative to jumps\\
	$\pi$:co-c.e. sets relative to jumps\\
	$A$ is arithmetical if $A \in \sum_n^0$ for some $n$.\\
	Oracle version $\sum_n^0(c)$ if $C \geq_\mathrm{T}R$\\
	$\exists y_1 < 27 \rightarrow$ we can rename it\\
	1.4.13: Let $n \geq 1$\\
	$A$ is $\sum_n^0 \iff A$ is c.e. relative $\phi^{(n-1)}$\\
	$\phi^{(n)}$ is $\sum_n^0$-complete\\
	$A \in \delta_2^0 \iff A$ is $\sum_2^0$ and $\pi_2^0$.\\
	$A$ is $\delta_n^0 \iff A \tr \phi^{(n-1)}$\\
	ex. $\{e:W_e$ is finite\}$=F\mathrm{in}\\
	e \in F \iff \left(\exists n_1 \in \N\right)\left( \forall n_2 > n1 \forall s \in \N \right)\\
	n_2 \notin W_{e,s}\\
	F\mathrm{in}$ is $\sum_2^0$-complete.\\\\
	Tot = $\{e:\dom(\phi_e)=\N\}$\\
	$e \in$ Tot $\iff (\forall y_1 \in \N)(\exists s \in \N)$
	\textbf{Missing the bit at the end}\\\\	
	\textbf{Dara:} High is opposite of computably dominated.\\
 	Set $A$ is \textbf{high} if $\phi^{11} \tr A'$.\\
 	A set $A$ is high$_n$ if $\phi^{(n+1)} \tr C^{(n)}$\\
 	\begin{theorem}[1.5.19]
 		$C$ is high $\iff$ some function $f$, $f \tr C$ dominates all computable functions.
 	\end{theorem}
	Going to construct $f$ function.
	\begin{proof}
		$\implies$ Want $f \tr$ that domaintes all comp.\\
		$A:= \{<e,x>:\phi_e(x)\halts\}$ is c.e. $\implies A \mor \phi'$ via $h$ comp.\\
		Let $f(x)=\max\{\phi_e(x):e \leq x$ and $h(<e,x>) \in \phi'\}$\\
		If $\phi_e$ is total, then $f(x) \geq \phi_e(x) \forall x \geq e$.\\
		Tot=$\{e:\phi_e$ total\} is $\pi_2^0$\\
		Tot $\mor N-\phi''\tr C'$\\
		By the limit lemma, there is a binary function $p \tr C$ s.th. for each $e$ $\lim_s p(e,s)$ exists and $\lim p(e,s)=1 \iff \phi_e$ is total.\\
		$Z$ is $\delta_2^0$ (missing a line here, or should this even be here?)\\
		
	\end{proof}

	\subsection*{Lecture 2020.03.06}
	Some sort of diag.
	Computable sets i.e. $\mathrm{deg}_T(\phi)$\\
	\\
	$C$ is high $\iff$ some function $f$, $f \tr C$ dominates all comuptable functions.\\
	$(\impliedby)$\\
	We will show that $\N \setminus \phi''=\{e:\Phi_e^{\phi'}(e)\uparrow \} \tr C'$.\\
	Note that $\Phi_e^{\phi'}(e)\uparrow \iff$ the computation at infinitely many steps i.e. no computation $\Phi[s]$ is stable.\\
	$\iff$ the partial computable function \[g(s) \simeq \mu t, t > s [\Phi_e^{\phi'}(e)[t]\uparrow]\ \mathrm{is\ total} \]
	\[e \notin \phi'' \iff \exists n_0 \forall n \geq n_0, \exists t \left[n \leq t \leq f(n)\ \mathrm{and}\ \Phi_e^{\phi'}(e)[t]\uparrow \right]  \]
	Since $t$ is bounded by $f(n)$ and $f \tr C$\\
	So $\N\setminus\phi''$ is $\sigma_2^0(C)$.\\
	Also $\phi'' \in \sum_2^0 \subseteq \sum)2^0(C)$.\\
	Therefore $\phi'' \tr C'$ by prop 1.4.14 relative to $C$.\\
	Prop: $A \in \delta_2^0 \iff A \tr \phi' \iff A$ and $\N \setminus A$ are c.e. in $\phi' \iff A \in \sum_2^0 \cap \pi_2^0$.\\
	
	\textbf{Coman}\\
	$Z$ is $\Delta_2^0$ if $\exists$ computable sequence of stong indices $(Z_s)_{s \in \N}\ \sth Z_s \subseteq [0,5]$ and $Z(x)=\lim 2s(x)$.\\
	Post Problem 1944\\
	Whether a c.e. set can be uncomputable and Turing incomplete.\\
	i.e. of there is a c.e. set $A\ \sth \phi \tr A \tr \phi'$\\
	Turing Incomparable Sets:\\
	Sets $Y,Z$. write $Y \mid_\mathrm{T}Z$ if $Y \not \tr Z \& Z \not \tr Y$\\
	\\
	Thm. 1.6.1\\
	There are sets $Y,Z \tr \phi''\ \sth Y \mid_\mathrm{T}Z$.\\
	\begin{proof}
		$Y \mid_\mathrm{T} Z$ this is equivalent to the conjunction of statements $R_i \forall i$, where \begin{align*}
			R_{2e} & :\exists n \neg Y(n)=\phi^2_e(n)\\
			R_{2e+1} & : \exists n \neg Z(m)=\phi^4_e(n)
		\end{align*}
	\end{proof} 

	\subsection*{Lecture 2020.03.11}
	\textbf{Simple}\\
	Is there a c.e. set that is neither computable nor Turing Complete?\\
	\begin{definition}
		c.e. $N \setminus A$....missing
	\end{definition}
	$N-A$ is not c.e. $\therefore$ $A$ is not computable.\\
	$A$ is so large that it meets each infinite c.e. set.\\
	By the padding lemma 1.1.3 obtain an infinite c.e. set $w$ of indices for the empty set $\sth\ \phi'' \cap W = \emptyset$.\\
	Build $A$=computable enumeration\\
	Divide the task into requirement.\\
	1.9) $S_i : \# W_i = \infty \implies W_i \cap A \neq \phi$ but also keeps $A$ co-infinite. (low, simple?)\\
	$s > -, A_0 = \emptyset$\\
	$A_{s-1}$ is finite.\\
	During stage $s$ we determine?\\
	$F \subseteq U$, $A_s=A_{s-1}\cup F$ each stage can be built using previous.\\
	$A_s$ is the value by the end of stage $s$. Has enumerated all elements in $F$ into $A$.\\
	$A$ c.e. set $U_sA_s$.\\
	$i < s$\\
	$A_{s-1} \cap W_{i,s-1}=\emptyset$, $\exists x \in W_{i,s}\ \sth\ x \geq 2i$.\\
	Want to guarantee $A$ coinfinite. $\therefore S_i$ acts (requirement taken care of).\\
	Act says in $A$ and $W_i$.
	If $W_i$ is infinite, infinite candidates gives that $S_i$ acts at most once?\\
	Cannot guarantee that for every finite $W_i$ this works.\\
	$\#A \cap [0,2e) \leq 0 \forall e\qquad A$ is co-infinite.\\\\
	There's a low simple set $A$.\\
	$J$ is (JUMP) machine taking oracle access to $A$ on input $e$. Because there's infinitely many stages, where\\
	Jump is halting problem relative to $A$.\\
	Don't have (because of bracket notation) oracle access to $A$, just $A$ up to $s-1$. This is because I'm currently constructing $A$.\\
	\textbf{Lowness requirement}:\\
	$L_e:\exists^\infty sJ^A(e)[s-1]\halts \implies T^A(e)\halts$\\
	Bit is kinda the same as $J_s^{A_s}(e)$.\\
	Can't say if it halts infinitely often it (something?).\\
	$A'(e)=\lim f(e,s),\qquad f= \begin{cases}
			1 & J^A(e)[s]\halts\\
			0 & \mathrm{otherwise}
	\end{cases}$\\
	$\Delta_2$ sets can change their minds as many times as they want but they will eventually settle.\\
	$A' \leq \phi'\qquad A$ is low. $L_e$ prevents $x < s$.
	On $L_e$: $A \upharpoonright$ use is stable. ($A=\star^\downarrow \mid_\mathrm{use}$??? big circle thingy space yoke)
	\[S_1 > L_1 > S_2 > L_2 \ddd L_e\]
\end{document}