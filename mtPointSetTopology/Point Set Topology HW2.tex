\documentclass{article}

\usepackage{amsmath}
\usepackage{amssymb}
\usepackage{amsthm}
\usepackage{pgfplots}
\usepackage{cancel}
\usepackage[a4paper,margin=1in]{geometry}
\usepackage{framed}
\usepackage{hyperref}

\title{MT434 Point Set Topology - Assignment 2}
\author{Ciarán Ó hAoláín - 17309103 - ciaran.ohaolain.2018@mumail.ie}

\let\nset\varnothing
\let\ddd\cdots

\newcommand{\sth}{\mathrm{s.th}\ }
\newcommand{\R}{\mathbb{R}}
\newcommand{\N}{\mathbb{N}}
\newcommand{\Z}{\mathbb{Z}}
\newcommand{\C}{\mathbb{C}}
\newcommand{\Q}{\mathbb{Q}}
\renewcommand{\H}{\mathbb{H}}
\newcommand{\quotient}[2]{{\raisebox{.2em}{$#1$}\left/\raisebox{-.2em}{$#2$}\right.}}
\newcommand{\abs}[1]{\left|#1\right|}
\newcommand{\norm}[1]{\left|\left|#1\right|\right|}
\newcommand{\cts}{cts. }
\newcommand{\ctsm}{\ \mathrm{cts.}\ }
\newcommand{\Int}[1]{\mathrm{int}(#1)\ }
\newcommand{\bdy}[1]{\mathrm{bdy}(#1)\ }
\renewcommand{\emptyset}{\varnothing}

\newcommand{\ba}{\mathcal{B}}

\newtheorem{theorem}{Theorem}[section]

\newtheorem{corollary}{Corollary}[theorem]

\newtheorem{lemma}[theorem]{Lemma}
\newtheorem{definition}[theorem]{Definition}
\newtheorem{stheorem}{Theorem}[theorem]
\newtheorem{slemma}[stheorem]{Lemma}
\newtheorem{sdefinition}[stheorem]{Definition}
\newtheorem{scorollary}[stheorem]{Corollary}
\theoremstyle{remark}
\theoremstyle{example}
\theoremstyle{examples}
\newtheorem*{remark}{Remark}
\newtheorem*{remarks}{Remarks}
\newtheorem*{example}{Example}
\newtheorem*{examples}{Examples}

\makeatletter
\newcommand{\skipitems}[1]{%
	\addtocounter{\@enumctr}{#1}%
}
\makeatother

\begin{document}
	\maketitle
	\begin{framed}
		\textbf{Notation:}\\
		I will denote the open ball of radius $r$ at point $p$ by $B_r(p)$.
		\[B_r(p) := \left\{ x \in \R^n : \norm{x-p} < r \right\}\]
		And also make use of the following shorthands:
		\[\Q^+:=\left\{q \in \Q\mid q > 0\right\}\]
	\end{framed}
	\begin{enumerate}
		\item Consider the countable set $\Q$. It is obvious that the cartesian product $\Q \times \Q$ is also countable.\\
		Now define the set $B$ as follows: \[B = \left\{\left(a,b\right) \mid a,b \in \Q \right\}  \]
		It is obvious that $\#B=\#\left(\Q \times \Q \right) = \#(\Q^2) = \left(\# \Q\right)^2$ (the set is countable).\\
		\textbf{Claim:} $B$ forms a basis for the standard (metric) topology on $\R$.
		\begin{proof}
			Let $I \in B$. Clearly $I=(a,b), a, b \in \R$.\\
			$I$ is then, by definition, an open interval in $\R$ w.r.t. the standard (metric) topology.\\
			$\therefore$ every set generated by $B$ is open in the toplogy.\\
			\\
			Conversely, if a set $I$ is open in $\R$ and $x \in I$, then $\exists (a,b) \subseteq I\ \sth x \in (a,b)$.\\
			$\therefore a < x < b$. We can now pick $u, v \in \Q\ \sth a < u < x < v < b$.\\
			Then we have that $x \in (u,v) \subseteq(a,b) \subseteq I$.\\
			$\therefore I$ is open relative to $B$.\\
			Thus we have that $B$ generates the standard (metric) topology on $\R$, proving the claim.
		\end{proof}
		This approach also works for $\R^n$, however. Consider the set $B'$ of open balls defined as follows: \[B' = \left\{ B_r(p) \mid p \in \Q^n, r \in \Q^+ \right\}\] This basis is also countable in the same way fashion as the case above for $\R$.
		\textbf{Claim:} $B'$ forms a basis for the standard (metric) topology on $\R^n$.
		\begin{proof}
			Let $X \in B'$. Clearly $X=B_r(p)$ for some $r \in \Q^+, p \in \Q^n$.\\
			The ball $X$ is, by definition, open in $\R^n$ w.r.t. the standard (metric) topology.\\
			$\therefore$ every set generated by $B'$ is open in the topology.\\
			\\
			Conversely, let $X$ be an open set in $\R^n$.\\
			Then we have that $\exists \underbrace{X'}_{\ni p} = B_{r'}(p') \in B'$, $r' \in \Q^+, p' \in \Q^n, X'\subseteq X\  \forall p \in X$.\\
			$\therefore X$ is open relative to $B'$.\\
			Thus we have that $B'$ generates the standard (metric) topology on $\R^n$, proving the claim.
		\end{proof}
		This shows that $B'$ is a countable basis for the standard (metric) topology on $\R^n$, as required.\\
		We can thus also conclude that $(\R^n,T_\mathrm{standard})$ is \textit{second countable}.
		
		\item 
		\begin{enumerate}
			\item Recall the definition of a basis:
			\begin{framed}
				\setcounter{section}{2}
				\setcounter{theorem}{2}
				\setcounter{stheorem}{0}
				\begin{sdefinition}
					If $X$ is a non-empty set, a \textbf{basis} is a family of subsets $\ba$ s.th. \begin{enumerate}
						\item $\forall x \in X, \exists B \in \ba\ \sth x \in B$
						\item if $x \in B_1 \cap B_2$, where $B_1,B_2 \in \ba$ then $\exists B_3 \in \ba\ \sth x \in B_3 \subseteq B_1 \cap B_2$
					\end{enumerate}
				\end{sdefinition}
			\end{framed}
			Let us verify that $\ba = \left\{\left(a,\infty\right),a \in \R \right\}$ fulfils these criteria on the set $\R$.\\
			\begin{enumerate}
				\item Let $\alpha  \in \R$. Clearly $\alpha - 1 \in \R$, $\alpha \in \underbrace{(\alpha - 1, \infty)}_{=B} \in \ba$, as required.
				\item Let $B_1=(a,\infty), B_2=(b,\infty) \in \ba$.\\
				Assume w.l.o.g. that $a < b$. Then $B_1 \cap B_2 = B_2$.\\
				$\therefore$ $x \in B_1 \cap B_2 \implies x \in B_2 \in \ba$.\\
				Then we can simply let $B_3 = B_2$, giving $\exists B_3 \in \ba\ \sth x \in B_3 \subseteq B_1 \cap B_2$ since $x \in B_2$.
			\end{enumerate}
			$\therefore$ we have that $\ba$ is a basis for a topology on $\R$.
			
			\item Recall the definition of continuity in this context:
			\begin{framed}
				\setcounter{theorem}{3}
				\setcounter{stheorem}{0}
				\begin{sdefinition}
					A map $f:(X,T) \to (Y, T')$ is \textbf{continuous} if $\forall V \in T'$ the pre-image $f^{-1}(V) \in T$
				\end{sdefinition}
			\end{framed}We have $T_\ba$ as follows: \[T_\ba=\left\{ I \subset \R \mid \forall x \in I , \exists B_x \in \ba\ \sth x \in B_x \subset I\right\}=\left\{ (a, \infty) , a \in \R \right\}\]
			\begin{enumerate}
				\item Let $V=(0,\infty) \in T_\ba$. Then $f^{-1}(V)=\{x:x^2 \in V\} = \{x \in \R : x^2 > 0\}=\R \setminus \{0\} \notin T_\ba$.\\
				$\therefore f$ is not \cts.
				\item Let $V=(a,\infty) \in T_\ba$.\\
				Then $g^{-1}(V)=\{x:e^x \in V\}=\{x \in \R: e^x > a\}=(\ln(\min\{0,a\}), \infty)$.\\
				So for any $a$ (i.e. for any $V \in T_\ba$), we have $g^{-1}(V)$ is the interval $(\ln(\min\{0,a\}),\infty) \in T_\ba$.\\
				$\therefore g$ is \cts.
			\end{enumerate}
		\end{enumerate}
		\pagebreak
		\item Recall the definition of the subspace topology:
		\begin{framed}
			\setcounter{section}{3}
			\setcounter{theorem}{1}
			\setcounter{stheorem}{0}
			\begin{sdefinition}
				For $A \subset (X,T)$, the \textbf{subspace topology} $T_A$ is \[T_A  = \left\{ U \cap A \mid U \in T \right\} \]
			\end{sdefinition}
		\end{framed}
		We have that $(X,T)$ is a topological space, $(Y,T_Y)$ is a subspace. This gives, by definition, that \[ T_Y = \left\{ U \cap Y \mid U \in T \right\} \]
		Furthermore, we have that $A \subset Y$.\\
		We need to show that the subspace topology $T_{Y_A}$ on $A$ as a subspace of $(Y,T_Y)$ is the same as the subspace topology $T_A$ on $A$ as a subspace of $(X,T)$.\\
		By definition, we have that \[T_{Y_A} = \left\{ U \cap A \mid U \in T_Y \right\} = \left\{ U \cap A \mid U \in \left\{ V \cap Y \mid V \in T \right\} \right\} \]
		We also have that \[T_A = \left\{U \cap A \mid U \in T \right\}\]
		To complete the proof, we need to show that $T_{Y_A} = T_A$.\\
		Let $X \in T_{Y_A}$. $X = U \cap A$, $U= V \cap Y$, some $V \in T$.\\
		$\therefore X=V \cap Y \cap A=V \cap A$ for some $V \in T$ since $A \subset Y \implies X \in T_{Y_A} \implies X \in T_A$.\\
		i.e. $T_{Y_A} \subset T_Y$.\\\\
		Let $X \in T_A$. $X=U \cap A$, $U \in T$. But $A=Y \cap A$.\\
		$\therefore X=U \cap Y \cap A, U \in T$. Note $U \cap Y \in T_Y$, $\therefore X=Z \cap A, Z \in T_Y \implies X \in T_{Y_A}$\\
		i.e.  $T_{Y_A} \subset T_Y$.\\
		$\mathbf{\therefore T_{Y_A}=T_A}$ as required.\\
		Also, not sure, but maybe can take $T_A = \{U \cap A \mid U \in T\} = \{U \cap Y \cap A \mid U \in T\}$\\
		$= \{Z \cap A \mid Z \in T_Y\}=T_{Y_A}$ (since $\{U \cap Y, U \in T\}=T_Y$).
		
		
		\item Let $Y' \subset Y$ be a closed subset of $Y$ w.r.t. $T'$. We need to verify that $h^{-1}(Y') \notin T\ \forall Y' \notin T'$.\\
		But $h^{-1}(Y')=(f^{-1}(Y')\cap A) \cup (g^{-1}(Y') \cap B)$. The two sides of this union are closed\\
		$\therefore$ the union is also closed.\\
		$\therefore h^{-1}(Y')$ is closed.\\
		$\therefore$ we have that $h:X \to Y$ is \cts, as required.
		
		\item We can do this by showing $\overline{A \times B}$ is open in $X \times Y$.\\
		Consider $(x,y) \in \overline{A  \times B} \iff x \notin A\ \mathrm{or}\ y \notin B$.
		This can rephrased as \[X \times Y \setminus A \times B = (X \setminus A) \times Y \cup X \times (Y \setminus B)\]
		We already know that $A$ is closed in $X$, therefore $X\setminus A$ is open in $X$. This gives that $(X \setminus A) \times Y$ is open in $X \times Y$. We can apply a similar argument to find that $X \times (Y \setminus B)$ is open.\\
		This gives that $\overline{A \times B}$ is the union of open sets, and is therefore open.\\
		$\therefore A \times B$ is closed, as required.
	\end{enumerate}
\end{document}