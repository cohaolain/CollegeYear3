\documentclass{article}

\usepackage{amsmath}
\usepackage{amssymb}
\usepackage{amsthm}
\usepackage{pgfplots}
\usepackage{cancel}
\usepackage[a4paper,margin=1in]{geometry}


\usepackage{hyperref}



\title{MT434P Point Set Topology - Summer Exam}
\author{Ciarán Ó hAoláín - 17309103}

\let\nset\varnothing
\let\ddd\cdots

\newcommand{\sth}{\mathrm{s.th}\ }
\newcommand{\R}{\mathbb{R}}
\newcommand{\N}{\mathbb{N}}
\newcommand{\Z}{\mathbb{Z}}
\newcommand{\C}{\mathbb{C}}
\newcommand{\Q}{\mathbb{Q}}
\renewcommand{\H}{\mathbb{H}}
\newcommand{\quotient}[2]{{\raisebox{.2em}{$#1$}\left/\raisebox{-.2em}{$#2$}\right.}}
\newcommand{\abs}[1]{\left|#1\right|}
\newcommand{\cts}{continuous\ }
\newcommand{\ctsm}{\ \mathrm{cts.}\ }
\newcommand{\Int}[1]{\mathrm{int}(#1)\ }
\newcommand{\bdy}[1]{\mathrm{bdy}(#1)\ }
\renewcommand{\emptyset}{\varnothing}
\renewcommand{\c}{\mathcal{C}}
\newcommand{\im}[1]{\mathrm{Im}\left(#1\right)}

\newcommand{\ba}{\mathcal{B}}


\newtheorem{theorem}{Theorem}[section]

\newtheorem{corollary}[theorem]{Corollary}

\newtheorem{lemma}[theorem]{Lemma}
\newtheorem{definition}[theorem]{Definition}
\newtheorem{stheorem}{Theorem}[theorem]
\newtheorem{slemma}[stheorem]{Lemma}
\newtheorem{sdefinition}[stheorem]{Definition}
\newtheorem{scorollary}[stheorem]{Corollary}
\theoremstyle{remark}
\theoremstyle{example}
\theoremstyle{examples}
\newtheorem*{remark}{Remark}
\newtheorem*{remarks}{Remarks}
\newtheorem*{example}{Example}
\newtheorem*{examples}{Examples}

\makeatletter
\newcommand{\skipitems}[1]{%
	\addtocounter{\@enumctr}{#1}%
}
\makeatother

\begin{document}
\maketitle

\begin{enumerate}
	\item \begin{enumerate}
		      \item Such an example cannot exist, since a subset equipped with the subspace topology has a set of open sets which is is a subset of the open sets of $X$.
		      \item This cannot occur. This is because $\R$ is connected. Therefore any mapping from it to the discrete space cannot be both \cts and non-constant. Like in the case of $f:\R \to Y$, $Y$ a set of $\geq 2$ elements (disconnected), any map which is \cts will be a constant map.
		      \item Such a topology is $\tau=\{\emptyset, \{1,2\}, \{3,4,5\}, \{1,2,3,4,5\}\}$.
		      \item This cannot occur, since any such spaces would be homeomorphic to the same objects.

	      \end{enumerate}
	\item \begin{enumerate}
		      \item The interior of $A$, int$A$ is \[\mathrm{int}{A}=\{a \in A \mid \exists U_a \in T, a \in U_a, U_a \subset A\}\]
		            Therefore we must have that $\mathrm{int}(A) \subset A \implies f(\mathrm{int}(A)) \subset f(A)$.
		      \item The first line gives that $f$ of the interior is a subset of $f$ on the set. The next line gives that this subset is necessarily open.\\
		            This then gives that it is the subset of $\Int{f(A)}$
		      \item Since $U \in T$, we have that $U=\Int{U}$, then $\Int{f(U)}\subset f(U)$ gives $f(U)=\Int{U}$.
		      \item This means that $f(U)$ must be open as above, giving that $f$ is an open map.
	      \end{enumerate}
	\item \begin{enumerate}
		      \item This is false. Consider $\{0\}$. We view the graph of $f$ as being equivalent to $\R^2$ Removing this point in $\R$ is an issue as it corresponds to a point in $\R^2$. This is because otherwise we would have that there is a homeomorphism from $f: \R \to \R^2, f|_{\R \setminus \{0\}} : \R \setminus \{0\} \to \R^2 \setminus \{f(0)\}$.\\
		            This would imply that $(f|_A)^{-1}$ is \cts. Since it is equal to $\R \setminus \{0\}$, it would mean $\R^2 \setminus\{f(0)\}$ is connected. Then $\R\setminus\{0\}$ is connected. But this is a contradiction, since $\R\setminus\{0\}=(-\infty,0) \cup (0,\infty)$.\\
		            Therefore such a homeomorphism cannot exist, and $f$ cannot be homeomorphic to $\R$.
		      \item This is true, the quotient seperates 0 and 1 here the same way the discrete topology would.
		      \item This is not true.\\
		            The set $\Q \cup \{\sqrt2,\pi\}$ is not compact, as it is neither bounded nor closed.
		      \item This is false, but the opposite is true.\\
		            Since path-connected $\implies$ connected (but not necessarily the converse), (i.e. connected component is a weaker condition) there may in fact be \textbf{more} connected components than path-components.
	      \end{enumerate}
	\item \begin{enumerate}
		      \item We have that $\Z, \emptyset \in S$. We now need to check (A) that any union of sets in $T$ belongs to $T$, and (B) that any intersection of finitely many sets in $T$ belong to $T$.\\
		            (A) is true, since larger values of $n$ the set $\{-2n, -2n + 1, -2n + 2, \ddd , 2n - 1, 2n, 2n + 1\}$ includes all the smaller sets (for smaller $n$). So the union of two sets with different $n$ is the same as a the set generated by the one with larger $n$.\\
		            (B) is also true, since for different values of $n$ the set  $\{-2n, -2n + 1, -2n + 2, \ddd , 2n - 1, 2n, 2n + 1\}$ intersected with another of different choice of $n$, the intersection simply consists of the set generated by the smaller value of $n$.
		      \item Since $\ba$ is a countable basis, we can label it as $\ba=\{B_1, B_2,B_3\ddd\}$.\\
		            Consider $n \in \N, n \leq \#\ba$, and then choose $x_n \in B_n$. Define $S=\{x_1,x_2,\ddd\}$ a set of these points.\\
		            We claim that $S$ is dense (and obviously countable in $X$), yielding the result that $X$ is seperable as follows:\\
		            Choose a point $a \notin S$, $a \in (X \setminus S)$. Suppose $a \in A$, $A$ is open. We want to show that $\exists s \in S\ \sth s \in A$ also.\\
		            By the definition of a basis, for each $x \in X \exists B_x \ \sth x \in B_x$.\\
		            Therefore for some $B_\alpha$, $B_\alpha \subset A$. But that means that $s$ must also be in $A$!\\
		            This gives that any open set containing $s$ also contains $a$.\\
		            This gives that $S$ is dense since its closure is the whole of $X$. Since $S$ is countable, this means that we have $X$ contains a countable dense subset.\\
		            $\therefore X$ is seperable, as required.
		      \item \begin{enumerate}
			            \item We can see this by verifying that the complement $X \setminus C$ is open.\\
			                  A map $f:(X,T) \to (Y,T')$ is \cts if $\forall V \in T'$ the pre-image $f^{-1}(V) \in T$.\\
			                  Consider $X\setminus C = \{x \in X \mid f(x) \neq g(x)\}$:\\
			                  Therefore $x_f = f(x) \neq x_g = g(x)$, i.e. they are two distinct points in the Hausdorff space.\\
			                  These are therefore in two disjoint open sets $U_f, U_g$.\\
			                  By the continuity of $f$ and $g$, we can see that $f^{-1}({U_f})\cap f^{-1}(U_g)=\emptyset$.\\
			                  Since the set is open, we get that $C$ is closed, as required.
			            \item If density holds, then we have that the interior of the complement of the set is empty. This gives is contradiction if we assume that $f\neq g$, therefore $C$ dense $\implies f=g$.
		            \end{enumerate}
	      \end{enumerate}
\end{enumerate}

\end{document}