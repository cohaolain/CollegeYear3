\documentclass{article}

\usepackage{amsmath}
\usepackage{amssymb}
\usepackage{amsthm}
\usepackage{pgfplots}
\usepackage{cancel}

\title{MT441P Group Theory 2}
\author{Ciarán Ó hAoláín}

\let\nset\varnothing
\let\ddd\cdots

\newcommand{\sth}{\mathrm{s.th\ }}
\newcommand{\R}{\mathbb{R}}
\newcommand{\N}{\mathbb{N}}
\newcommand{\Z}{\mathbb{Z}}
\newcommand{\C}{\mathbb{C}}
\renewcommand{\H}{\mathbb{H}}
\newcommand{\quotient}[2]{{\raisebox{.2em}{$#1$}\left/\raisebox{-.2em}{$#2$}\right.}}
\newcommand{\abs}[1]{\left|#1\right|}
\renewcommand{\do}{\cdot}

\newcommand{\stab}{\mathrm{stab}}
\newcommand{\orb}{\mathrm{orb}}
\newcommand{\gp}{\mathrm{gp}}
\newcommand{\subgp}{\ \mathrm{subgp.}}
\newcommand{\tsubgp}{\triangleleft}
\newcommand{\tsupgp}{\triangleright}
\newcommand{\aut}{\mathrm{Aut}}


\newtheorem{theorem}{Theorem}[section]
\newtheorem{corollary}{Corollary}[theorem]
\newtheorem{lemma}[theorem]{Lemma}
\theoremstyle{definition}
\newtheorem{definition}{Definition}[section]
\theoremstyle{remark}
\newtheorem*{remark}{Remark}
\theoremstyle{example}
\newtheorem*{example}{Example}

\makeatletter
\newcommand{\skipitems}[1]{%
	\addtocounter{\@enumctr}{#1}%
}
\makeatother

\begin{document}
	\maketitle
	
	\section{Lecture 2 (19.09.27)}
	\subsection{Free Groups}
	\begin{definition}
		Let $S$ be a set.\\
		A free group over $S$ generated by S is a pair ($(F(S),j))$ with basis $S$.\\
		$F(S)$ a group, $j:S \to F(S)$ injective $\sth \forall \alpha:S \to G, G$ a group, there is a unique group homomorphism $\bar{\alpha}:F(S) \to G$ "extending" $\alpha$.\\
		i.e. $\bar{\alpha\cdot j = \alpha}$		\\
		Set $S \overbrace{\to}^\alpha G$ a group.\\
		$S \overbrace{\to}^j F(S)$ a group.\\
		$F(S) \overbrace{\to}^{\exists!\bar{\alpha}} G$ a group.
	\end{definition}
	
	
	\begin{theorem}
		For every set $S \exists $ a free group $(F(S), j)$ generated by $S$.\\
		Free groups generated by $S$ are unique up to isomorphism under $S$.\\
		$S \overbrace{\to}^j F(S)\qquad
		S \overbrace{\to}^{j'} F'(S)\qquad
		F(S) \overbrace{\to}^\eqsim F'(S)$
		
		i.e. if $(F(S),j),(F'(S),j')$ are free groups generated by $S$, then there is a unique isomorphism $\phi:F(S) \to F'(S) \ \sth \phi \cdot j = j'$
	\end{theorem}

	\begin{proof}
		Let $S$ be a set and $i:S \to \bar{S}$ bijective,\\
		$\bar{S} \cap S = \nset, S=\{a,b,c\}, \bar{S}=\{\bar{a},\bar{b}, \bar{c}\}.\\
		T=S \cup \bar{S}, i(s)=\bar{s},\ i(\bar{s})=s$\\
		A word in $T$ of length $n \in N_0$ is \[
		\begin{cases}
			1 & $if\ $ n=0\\
			\{1,2\ddd,n\} \to T & $if\ $ n>0, (t_1,t_2,\ddd,t_n)
		\end{cases} \]
		
		Equivalence relation:
		$(t_1 \ddd t_l \ t \ \bar{t}\  t_{l+1} \ddd t_n ~ (t_1 \ddd t_l t_{l+1} \ddd t_n)$\\
		Reduced word. Shortest in equiv. class. No subword $(t,\bar{t})$. A reduced word has no two cancelling elements adjacent.\\
		Concatenation:
		$s=(t_1 \ddd t_n), v=(S_1 \ddd S_m)\quad
		w \star v = (t_1 \ddd t_n \ s_1 \ddd s_n)\\\
		F(S)=\{$reduced words in $T\}$. Group multiplication is $\star$.\\
		$(F(S),\star)$ is a group.
		Neutral element: 1, empty word.\\
		Inverse of $(t_1 \ddd t_n)$ is $\bar{t_n} \ddd \bar{t_1}$\\
		Associativity:
		\[(t_1 \ddd t_n) \star ((s_1 \ddd s_m) \star (r_1 \ddd r_k))\\
		=((t_1 \ddd t_n) \star(s_1 \ddd s_m)) \star_1 \ddd r_k\]
	
		The Cayley graph of a free group is a tree.
		
		Given a group and multipication, you should be able to find the neutral element.
		
		Group structure: 1, "empty word" neutral element.
		
		$s_1 \ddd s_n \star r_1 \ddd r_m = s_1 \ddd s_n r_1 \ddd r_m$.
		
		To prove associativity, we can represent as a set that we already know to be associative.\\
		$\C = \{a+ib\}$ has a (matrix) multiplication $
		\begin{pmatrix}
			a & -b\\
			b & a
		\end{pmatrix}
		$
		
		Associativity of $\star$ with the "Van Der Waerden trick".\\
		\begin{align*}
			S \times F(S) & \to F(S)\\
			(s,s_1 \ddd s_n) \mapsto \begin{cases}
				s \ s_1 \ddd s_n & s \neq \bar{s_1}\\
				s_2 \ddd s_n & s = \bar{s_1}
			\end{cases}
		\end{align*}
		Defines a map \begin{align*}
			S & \to^\phi S_{F(S)}\\
			S & \mapsto (\Phi \mapsto \bar{\phi}(s, \Phi))
		\end{align*}
		
		extending to a homomorphism \begin{align*}
			F(S) & \to \Phi(s, \sigma) S_F(s)\\
			a \star b &\mapsto \phi(a) \dotso \Phi(b)
		\end{align*}
		
		$S_{F(S)} = $ symmmetric group (permutations) of set $S$.
		\end{proof}
	
		\begin{lemma}
			For every word $w+s_1 \ddd s_n \in F(S)$ there is a unique reduced word equivalent to $w$. 
		\end{lemma}
	
		\begin{proof}
			\begin{align*}
			w \to w_1 \to w_2  & \to \ddd \tilde{w} & (\mathrm{elementary\ reduction}).\\
			w \to w_1' \to w_2' & \to \ddd \tilde{w}' & \null\\
			\end{align*}
			$(s_1 \ddd s_l \ s \ \bar{s} \ s_{l+1} \ddd s_n) \to (s_1 \ddd s_l \ s_{l+1} \ddd s_n)$ reduces length by 2.\\
			
			By induction assume $w_1 \neq w_1'$.\\
			\textbf{2 cases:}
			\begin{enumerate}
				\item $w=(s_1 \ddd s_l \ s \ \bar{s} \ s \ s_{l+1} \ddd s_n)$ overlap $w_1 = w_1'$.
				\item $w = (s_1 \ddd s_l \ s \ \bar{s} \ s_{l+1} \ddd s_m \ t \ \bar{t} \ s_{m+1} \ddd s_n)\\
				z=(w_1 \ddd s_l \ s_{l+1} \ddd s_m \ s_{m+1} \ddd s_n) \to \ddd \to q$, reduced.
				
			\end{enumerate}
		
			By induction, since $\tilde{w}, q$ are reductions of $w_1, |w_1|=|w|-2, \tilde{w}=q = \tilde{w}'$.\\
			By induction, since $\tilde{w}', q$ are reductions of $w_1', |w_1'|=|w|-2, \tilde{w}=q = \tilde{w'}'$.
		\end{proof}
	
		\begin{definition}[Presentations]
			Let $G$ be a group, $S \subset G$ generating $G$.\[ \forall g \exists s_1\ddd s_n,e_1 \ddd e_n : g = s_1^{e_1} \ddd s_n^{e_n} \iff S \subset H \subset G \implies H = G\]\\
			$\ker \alpha \triangleleft F(S) \to^\alpha G$ ($\alpha$ unique, surjective).\\
			$S \to ^ j F(S)$, $S \to G$
		\end{definition}
		\begin{proof}[Proof. $\alpha$ is surjective\\]
			Othewise $\alpha(F(S)) \subset G$ would be a non-trivial subgroup of $G$ containing $S$.
		\end{proof}
		By Homomorphism Theorem: $\bar{\alpha} : $ \\
		% $\quotient{F(S)}{\ker \alpha} \to^\cong G$.
		\begin{definition}
			Let $G$ be a group and $R \subset G$.\\
			\begin{enumerate}
				\item $H \subset G$ is generated by $R$ if \[H=\bigcap_{X \subset G\ \mathrm{subgp.}, X \supset R} \iff \forall Q \subset_{\subgp}  H, Q \supset R, Q=H \]
				$ \iff H = \{$ products of elts. in $R\}$
				\item $H \subset G$ is normally generated by $R$/normal closure of $R$ if 
				$ H=\bigcap_{X \triangleleft G, R \subset X} X = \{$ conjugates of prod's (by elements of $g$ in $R ) \}$
				\[ g^{-1}(r_1^{e^1} \ddd r_l^{e_l}), r_i \in R, e_i \in \Z, g \in G \]
				If $R$ normally generates $\ker \alpha$ then write \[<S|R>:=\quotient{F(S)}{\ker\alpha} \cong G \]
				Where $S$ are the generators and $R$ are the relators. Every group has a permutation.
			\end{enumerate}
			\begin{example}
				\[ G=S_n = \{ \alpha : \{1 \ddd n \}  \looparrowleft | bijective \} \]\\
				$R=\{(1,2)\}$ gp. generated by $R$ is $\{id, (12) \}$\\
				Normal closure of $R$ $\supset \{id,(12),(13)=(23)^{-1}(12)(23),(p,q) \}=S_n,\quad p,q  \in \{1 \ddd n\}$
			\end{example}
		\end{definition}
	
		\begin{theorem}
			$(L_n . S_n )$ two permutations are conjugate if they contain cycles of same length.\\
			\textbf{Explanation missing...}
		\end{theorem}
		
		\begin{example}
			\begin{align*}
				1 & \cong <x|x>=<1>\\
				\Z_2 & \cong <x|x^2> = <x|x^2=1>\\
				\Z_6 & \cong <x|x^6> \cong <a,b|a^2,b^3,aba_{-1}b^{-1}>
			\end{align*}
			\begin{align*}
				1=x^6 & \mapsto a^6 . b^6 = 1\\
				x^4 & \mapsto a^4.b^4 = b\\
				x^3 & \mapsto a^3.b^3 = a
			\end{align*}
		\end{example}
	
		Groups of order $6$:\\
		\begin{align*}
			\Z_6 & \cong \Z_2 \times \Z_3\\
			<x|x^6> & \cong <z,y|z^2,y^3,zyz^{-1}y^{-1} \iff zyz^{-1}y^{-1}=1>\\
			\\
			S_3 & \cong <x,y | x^2,y^2,(xy)^3 ,(yx)^3> \leftarrow F\{x,y\}\\
			x & \mapsto (12)\\
			y  & \mapsto (13)\\
			(13) & = (23)(12)(23)\\
			xy & \mapsto (12)(23) = (132)\\
			yx \mapsto (23)(12) = (123)
		\end{align*}
		Listing elements: \[1,x,y,xy,yz,xyx,yxy \ddd \]
		\begin{align*}
			xyx=(y \times y)^{-1}] & \iff xyxyxy=1=(xy)^3\\
			(xyx)^2=xyxxyx=1 & \implies xyx=yxy
		\end{align*}
		
		Groups of order $7$:\\
		\[
			\Z_7 \cong <x|x^7>
		\]
		
		Groups of order $8$:\\
		\begin{align*}
			\Z_8 & = <x|x^8>\\
			\Z_4 \times \Z_2 & = <x,y| x^4,y^2,xyx^{-1}y^{-1}>\\
			\Z_2 \times \Z_2 \times \Z_2 & = <x,y,z|x^2,y^2,z^2,xyx^{-1}y^{-1},xzx^{-1}z^{-1},yzy^{-1}z^{-1}>
		\end{align*}
		
		Symmetries of $\square\ (D_4 \subset S_4)$ (dihedral group).
		\[D_4 \cong <x,y| x^4,y^2,x^3yx^{-1}y>\] 
		\[ x \mapsto R, y \mapsto \sigma \]
		
		$Q_8$\\
		\[ \H \cong \R^3 \cong \C^2 = \left\{ \begin{pmatrix}
			a & -b\\
			\bar{b} & \bar{a}
		\end{pmatrix} \mid a,b \in \C \right\}, \det=a \bar{a}+b \bar{b} = |a|^2+|b|^2 \]
		\[ \R^2 \cong \C = \left\{ \begin{pmatrix}
			a & -b\\
			b & a
		\end{pmatrix} \mid a,b \in \R \right\}, \det = |a+ib|^2=a^2+b^2 \]		
		\[ q=(xyzu) \in \R^4 \qquad ||q||^2 = \det q = x^2 + y^2 + z^2 + u^2 \]
		If $q,p \in S^3$, then $\det pq = \det p \det q = 1$.
		$Q_8 \subset S^3 \subset \H^\times$ is a subgroup.\\
		$Q_8$ is a group generated by $\{ i,j\}, k=ij$\\
		\[\cong <x,y|x^4,y^4,x^2y^2,(xy)^4>\]\\
		$x \mapsto i, y \mapsto j$.


		\section*{Lecture 7 (19.10.15)}
		Let the group $g$ act on the set $X$, $x_0 \in X$
		\begin{align*}
		G_{x_0} & \to G & \to > G_{x_0} \to X\\
		\stab g & \mapsto g \gp & \leftarrow^s & \orb
		\end{align*}
		
		\section*{Lecture 8 (19.10.18)}
		\begin{definition}[Group actions]
			$G \overset{\alpha}{\to}S(X)$ gp. hom.
			(left actions)
			\begin{align*}
				g & \mapsto (x \mapsto gx)=(\alpha(g))x\\
				hg & \mapsto (\alpha(h) \circ \alpha(g):x \to h(g(x)))
			\end{align*}
			\begin{align*}
			G \times X &\overset{\alpha}{\to} X\\
			(g,x) & \mapsto gx=(\alpha(g))x
			\end{align*}
			\[X=\bigcup_{x_0 \in X}Gx_0=\bigcup_{x_0 \in S}Gx_0\]
			$S$ section, a set containing one element from each orbit.
			\[\abs{Gx_0}.\abs{G_{x_0}} = \abs{G} \]
			$\{g \in G|gx_0=x_0\}=G_{x_0}$ stabiliser of $x_0$
		\end{definition}
	
		\begin{definition}[Conct formula]
			\begin{align*}
				\abs{X}=\sum_{x_0 \in S} \abs{Gx_0}&=\sum_{x_0 \in S}\abs{G:G_{x_0}}\\
				&=\sum_{x_0 \in S}\frac{\abs{G}}{\abs{G_{x_0}}}
			\end{align*}
		\end{definition}
		\begin{definition}[Special case (class formula)]
			$X=G, G \circlearrowleft G$ conjugation.
			\begin{align*}
				G & \to S(G)\\
				g & \mapsto \left( \underbrace{x}_{\in G} \mapsto \underbrace{g \times g^{-1}}_{\mathrm{conj.}\ x\ \mathrm{by}\ g}=x^y \right)
			\end{align*}
		\end{definition}
		\begin{definition}[Orbits of conjugation]
			Conjugation class\\
			Conjugacy class of $x \in G$ \[ \{g \times g^{-1} | g \in G \} \]
			Stabiliser of $x \in G$ for conjugation is \[G_x=\{g \in G | g \times g^{-1}=x\}\]
			$=C_G(x)$ centraliser of $x$ in $G$.\\
			=\{elts. of $g$ commuting with $x$\}\\
			
			Then \[\abs{G}=\sum_{x_0 \in S}\abs{G.C_G(x_0)}=\abs{Z(G)}+\sum_{x_0 \in S \setminus Z(G)}\abs{G.C_G(x_0)}\]
			$x_0 \in S \setminus Z(G)$ is the non-trivial conjugacy classes.
		\end{definition}
	
		Center of $G$,$Z(G)$
		\begin{align*}
		Z(G)&=\{x \in G|C_G(x)=G \} & \forall g\in G. gx=xg\\
		& = \{x \in G | \forall g \in G\quad g \times g^{-1}=x \}
		\end{align*}
		
		\begin{example}
			$p$ a prime, $G$ a p-group (i.e. $\#G=p^n$, some $n$).\\
			Then by the class formula \[p^n=\#Z(G)+\sum_{x_0 \in S \setminus Z(G)} \frac{\#G}{\#C_G(x)}\]
		\end{example}
		\pagebreak
		\section*{Lecture 9 (19.10.21)}
		Class formula:\\
		$G \circlearrowleft G$ ($G$ acting on itself)\\
		by conjugation.\\
		\[\#G=\#Z(G)+\sum_{g \in a} \abs{G:C_G(g)}\]
		$G \circlearrowleft X$ \[ \#G.x \circ \#G_x = \#G \]
		\begin{theorem}[Sylow Theorem]
			$G$ finite, $\abs{G}=p^\alpha.m$.\\
			$p \times m (P \subset G, \abs{P}=p^\alpha$ is acalled $p-$sylow subgp.)
			\begin{enumerate}
				\item $\#p$-Sylow subgroups of $G$ is $\equiv 1 \mod p ( = \abs{G:N_G(p)})$\\
				(in particular, there exists a $p-$Sylow subgroup.
				\item if $Q \subset G$ is a $p$ group ($\#Q=p^\beta,\quad 0 \leq \beta \leq \alpha)$\\
				then $\exists g \in G:gQg^{-1}\subset P$
				\item If $Q,P$ are Sylow-subgroups, then $Q=gPg^{-1}$ for some $g \in G$.
			\end{enumerate}
		\end{theorem}
		\begin{proof}
			$S=\{A \subset G | \#A=p^\alpha \}, S \circlearrowleft G$\\
			$gA=\{ga|a\in A \}$\\
			For $a \in S: G_A=\{g \in G |gA=A \}\qquad \forall a \in A:ga \in A \quad \circlearrowleft A$ (free)\\
			$G$ acts on $G$ by left multiplication, freely\\(no element fixes an element, $gx=x \implies g =1, \quad g,x \in G $)\\
			$p^\alpha=\#A=\abs{\underbrace{G_A.A}_{\subset G}}=|$orbits of $G_A$ on $A|$ . $G_A$.\\
			$\abs{\{gA|g\in G \}}=\abs{G.G_A}=mp^{\alpha-\beta}$\\
			IF $\abs{G.A}=m$ then $\abs{G_A}=p^\alpha$ hence $\#\{$orbits of $G_A$ on $A\}=1$\\
			$G_A.a=A$ for some $a \in A$\\
			$a^{-1}G_A.a=a^{-1} \in G.A$.\\\\
			CONVERSE if $G.A$ contains a group $H$ (a Sylow-gp) then $\abs{G.A}=\abs{G:H}=m$\\
			$\begin{pmatrix}
				p^\alpha m\\
				p^\alpha
			\end{pmatrix}=\abs{S}=\sum|G|$ (orbits of size $m) +\sum \abs{G} $(bigger orbits)\\
			$=\#$ Sylow subgroups$.m+m.p.r$, some $r$.\\
			CLAIM: \[\quotient{\begin{pmatrix}
				p^\alpha m\\
				p^\alpha
			\end{pmatrix}}{m}=1 \mod p\]\\
			PROOF: Take $G= \Z_{p^\alpha m}$, $G$ has exactly one $p-$Sylow subgroup
		\end{proof}
	\pagebreak
	\section*{Lecture 2019.11.11}
	In last \textbf{missed} lecture, proved that group of order 56 has 7 single subgroups.\\
	$|G|=56=7 \cdot$
	\begin{align*}
		n_7 = 8 & & p_7 \cong \Z_7\\
		n_7 = 8 & & p_2 \triangleleft G\\
		& & & |P_2|=8,p_7=...:
	\end{align*}
	Abelian, $D_4, Q_2, (\Z_8, Z_4 \times \Z_2, \Z_2 \times \Z_2 \times \Z_2)$.\\
	$G$ need not be abelian, can be \textbf{semidirect product}.\\
	\\
	$A$ abelian $=XY, X \cap Y=1 \implies A \cong X \times Y$\\
	If $N \triangleleft G, Q \subset G$, then $QN=\{qn|q \in Q, n \in N\}$ is a subgroup.\\
	\begin{definition}[Aut]
		\[\aut G = \{ \phi G \to G | \phi\ \gp\ \mathrm{iso} \}\]
	\end{definition}
	\begin{definition}
		Let $N,Q$ be groups ,and $p.Q \rightarrow \aut N$. Then the semidirect product of $N$ with $Q$ w.r.t. $p$ is \[ N \times_p Q = (\N \times Q, \star) \] where $(n,q) \star (m,p) = (n\mathbf{m^q},qp)= (n[p(q)](m),qp)$.\\
		$m^q=(p(q))(m)$ ? 
	\end{definition}

	A group $G$ is called simple if 1 and $G$ are its only normal subgroups.\\
	If $G$ is not simple then 1 $\neq N \triangleleft^{\neq} G$\\
	\begin{align*}
		1 & \to N  & \to_\mathrm{inj.} G& &\to_\mathrm{surj.}\quotient{G}{N} & \to 1\\
		& \downarrow f & \downarrow F & &   \downarrow q \\
		1 & \to \hat{N} & \to \hat{G} & & \to \hat{Q} & \to 1
	\end{align*}
	
	\begin{lemma}[5-Lemma]
		If $f$ and $q$ are isomorphismsms, then $F$ is isom.
	\end{lemma}

	Claim $G$ is a semidirect product.\\
	\begin{lemma}
		$N \triangleleft G \supset Q, N \cap Q = 1, NQ=G$, then $G \cong N \times_\mathrm{conj.} Q$.\\
		($p(q)n=qnq^{-1}$	)
	\end{lemma}

	\section*{Tutorial}
	\begin{enumerate}
		\skipitems{3}
		\item $Z \subset G$ central.\\
		$N \triangleleft G, s:\quotient{G}{N} \to G$ gp. hom.\\
		$\quotient{G}{N} \to_{id} \quotient{G}{N}\\
		G \to_\pi \quotient{G}{N}$.\\
		\\
		$1 \to N \triangleleft G \to_\pi \quotient{G}{N} \to 1\\
		\quotient{G}{N} \to s\\
		G \cong N \times| \quotient{G}{N}$ \\
		\\
		$N \times|_pQ \cong N \times Q $ if $p.Q \to \aut(N)$ is trivial.
		\item $H \subset G \ \forall p, p| \abs{G} \exists P_p \subset G$ a Sylow subgp.. $P_p \subset H \implies H=G$.\\
		Let $p|\abs{G}, p^k|\abs{G}$, $k$ maximal. Let $P_p \subset G, \abs{P_p}=p^k$ be in $HP_p \subset H \implies p^k | \abs{H}$.\\
		If $p^k|\abs{G}$ then $p^k | \abs{H} $ but $\abs{H} \abs{G}$\\
		$\abs{G}=p_1^{k_1}\cdot p_2^{k_2} \ddd p_r^{k_r} \implies $ they divide $\abs{H}$, $p_1^{k_1} \ddd p_r^{k_r} | \abs{H}$
		Warning: $\exists A,B,C \subset G$, $A \cap B = 1 = B \cap C = A \cap C$\\
		But $A \subset BC$\\
		$G = \Z_2 \times \Z_2$\\
		$A=\Z_2 \times 0\\
		B=0 \times \Z_2\\
		C=\{(0,0),(1,1)\}$
		\item $G$ finite, $n_p=1 \forall p | \abs{G}$\\
		$\implies G \cong P_1 \times P_2 \times \ddd \times P_r,P_n \ddd P_r$ the Sylow subgroup.
		\begin{proof}
			$P_1\cdot P_2 \ddd P_r \subset G$ is a subgroup, containing a Sylow subgroup of $G$ for each prime.\\
			Since $P_i \triangleleft G \forall i=1 \ddd r $ by assumption (uniqueness)\\
			$P_iP_j=P_jP_i,\quad x_1 \ddd x_r = x_r' \ddd x_1', x_i \in P_i, x_i' \in P_i\\
			N \triangleleft G$ normal, $x \in G$\\
			$xN=Nx, \forall n \in N \exists n_x:xn=n^xx,n^x=xnx^{-1}$.\\
			By (5), $P_1P_2 \ddd P_r=G \leftarrow^\phi P_1 \times P_2 \times \ddd \times P_r$\\
			$x_1x_2\ddd x_r \leftarrowtail (x_1 \ddd x_r)?$
		\end{proof}
		$\phi$ is a gp. hom.\\
		\begin{align*}
			\phi((x_1 \ddd x_r)(y_1 \ddd y_r))&=\phi(x_1y_1,x_2y_2, \ddd x_ry_r)\\
			&=x_1y_1x_2y_2 \ddd x_r y_r\\
			\phi(x_1 \ddd x_r) \cdot \phi (y_1 \ddd y_r &= x_1 x_2 x_3 \ddd x_r y_1 \ddd y_r)
		\end{align*}
		\begin{lemma}
			If $a,b \in G, a \in N \triangleleft G, b \in M \triangleleft G, n \cap M = 1\\
			$then $ab=ba, aba^{-1}b^{-1}=1$
		\end{lemma}
		\begin{proof}
			$N \ni aba^{-1}b^{-1} \in M$, hence $aba^{-1}b^{-1} \in M \in M \cap N = 1$.
		\end{proof}
		\item $G \circlearrowleft X, H$ transitive, $H \subset G$\\
		$\implies G = HG_x, x \in X$.
		\begin{proof}
			Let $g \in G, x \in X$. Then $\exists h \in H \sth gx=hx, h^{-1}gx=x\\
			h^{-1} \in G_x,g \in hG_x$
		\end{proof}
	\end{enumerate}

	\begin{lemma}[5-Lemma]
		\begin{align*}
			A &&& \to & B &&& \to & C &&& \to & D &&& \to & E\\
			\cong &&& \# & \cong &&& \# & \downarrow F &&& \# & \cong &&& \# & \cong  \\
			X &&& \to & Y &&& \to & Z &&& \to & W &&& \to & U\\
		\end{align*}
		\begin{proof}
			Claim that $F$ is isomorphism.\\
			\begin{align*}
				F(u)&=
			\end{align*}
		\end{proof}
	\end{lemma}

	\section*{Lecture 2019.11.15}
	\[\abs{G} = 2n, 2 \nmid n \implies \exists u \triangleleft^2 G \]
	
	\begin{proof}
		$\exists \tau \in G, \abs{\tau} =2$.\\
		
		$\begin{matrix}
		G & \to & S_{2n}\\
		\triangledown_2 & & \triangledown_2\\
		u & \to & A_{2n}\\
		\end{matrix}$,% \begin{cases}
		%	action of $G \circlearrowleft G$ by left multiplication\\
		%	\\
		%	$u=p^{-1}(A_{2n}) \triangleleft G$, index=2.
		%\end{cases}\\
		sgn($\rho(\tau))=(-1)^n=-1$\\
		\[ \forall \tau g \neq g \implies p(\tau) = (\ )(\ ) \ddd (\ )\] only 2-cycles, $n$ of them.
	\end{proof}

	\[\abs{G}=36 \implies \notag G\ \mathrm{not\ simple}\]
	\begin{proof}
		$G-2^2\cdot 3^2\\
		n_3=1 \mod 3, n_3 \mid 4, n_3=1 \mathrm{\ or\ } 4\\
		n_3 = 1 \implies $ 3-Sylow subgroup is unique, hence normal.\\
		$n_3=4$, Let $P, \tilde{P}$ 2 different 3-Sylow subgroups, $\abs{P}=\abs{\tilde{P}}=3^2$, $P,\tilde{P}$ abelian.\\
		\begin{enumerate}
			\item $P \cap \tilde{P}=1$ for every pair $P, \tilde{P}$ of 3-Sylow subgroups.\\
			$4 \cdot 8 + 1 $ elements. in 3-groups, $4 \cdot 8=32$ elements of order 3 or 9.\\
			$\implies 4$ elements left, are 2-Sylow subgroups $P_2, P_2 \triangleleft G$.
			\item $\exists P,\tilde{P}$ 3-Sylow subgroups, $P \cap \tilde{P} =Q\cong \Z^3\\
			\underbrace{C_G(Q)}_{\abs{\ }=3^2 \cdot 2^l, \mathrm{some\ } l} \supset P \cup \tilde{P}$\\
			$C_G(Q)$ contains at least 4 3-Sylow subgroups\\
			$C_G(Q)$ contains all 3-Sylow subgroups of $G$.\\
			$P \lneq C_G(Q) < G$.\\
			If $C_G(Q) \neq G$, then $C_G(Q) \triangleleft G$, group of index 2.\\
			If $C_G(Q) = G$, then $Q \triangleleft G$.
		\end{enumerate}
	\end{proof}

	If $\abs{G} = p^2, p, 1$ then $G$ is abelian ($p$ prime).\\
	If $\abs{G}$=$p^3$, $G$ must not be abelian, but has non trivial center.\\
	$n \leftarrow \underbrace{\quotient{G}{Z(G)}}_\mathrm{abelian} \leftarrow G{p^3} \leftarrow \underbrace{\mathrm{abelian}} \leftarrow 1$

	\begin{theorem}
		\begin{enumerate}\ 
			\item A p-gp has nontrivial centre.
			\item p groups are nilpotent.
		\end{enumerate}
	\end{theorem}
	\begin{proof}
		\begin{enumerate}
			\item Let $P$ be a $p$ group and look at its action on $P$ by conjugation.\\
			Class-formula:\\
			$(P=\cup$ conjugacy classes)\\
			$[p]=\{gpg^{-1}|g \in P \} = \quotient{P}{C_p(p)}\\
			\abs{P}=\abs{Z(P)}+p\cdot$ something
		\end{enumerate}
	\end{proof}

	\begin{definition}
		A group $G$ is nilpotent if there is a finite sequence $\{G_i \ \mid i=1 \ddd k\}$ of normal subgroups so that \[ G=G_0 \triangleright G_1 \triangleright \ddd \triangleright G_{k-1} \triangleleft G_k = 1 \] so that for all $j=0 \ddd k-1$
		\[ \quotient{G_j}{G_{j+1}} = Z \left(\quotient{G}{G_{j+1}}\right) \]
		which is in particular abelian.
		\[
		\begin{matrix}
			1 & \to & G_{k-1} & \to^Z &\to &G_{k-2} &\to & \quotient{G_{k-2}}{G_{k-1}} &\to &1
		\end{matrix}
		\]
	\end{definition}

	\section*{Lecture 2019.12.02}
	\textbf{HW Q.14}
	\[G=\left\{ \begin{pmatrix}
	1 &&& \star\\
	& 1\\
	&& 1\\
	0&&& 1
	\end{pmatrix} \mid \star \in \Z_2 \right\}\]
	Lower central series: $\begin{matrix}
			G & \tsupgp & [G,G] & \tsupgp& [G,G[G,G]]\\
			\downarrow\\
			\quotient{G}{[G,G]}
	\end{matrix}$
	$V=k^n.e_1 \ddd e_n$ basis.\\
	$E_{k,l}$ is identity with 1 at $k,l$.\\
	\begin{align*}
		E_{k,l}&=e+(e_j \mapsto e_l)\\
		E_{k,l}(e_j)&=\begin{cases}
			e_j & j \neq k\\
			e_j + e_l & j=k
		\end{cases}\\
		E_{k,l}E_{i,j}&=(e+(e_k \to e_l))(e+(e_i \to e_j))\\
		&=e+(e_i\to e_j)+(e_k \to e_l)+\begin{cases}
			0 & j \neq k\\
			e_i \to e_l & j=k
		\end{cases}\\
		E_{i,j}E_{k,l}&=(e+(e_i \to e_j))(e+(e_k \to e_l))\\
		&=e+(e_k\to e_l)+(e_i \to e_j)+\begin{cases}
			0 & l \neq i\\
			e_k \to e_j & l=i
		\end{cases}
	\end{align*}
	If degree $E_{k,l},E_{i,j}=d$, $k \geq l+d$, $i \geq j+d$.\\
	$[E_{k,l},E_{j,i}]=$1+some 2d-shift has degree $2d$.\\
	
	\[ \left\{\begin{pmatrix}
		1 & a & b & c\\
		& 1 & d & e\\
		& & 1 & f\\
		0 & & & 1
	\end{pmatrix} \mid a,b,c,d,e,f \in \Z_2 \right\} G \ \tsupgp \ G' \subset \begin{pmatrix}
		1 & 0 & \star & \star\\
		& 1 & 0 & \star\\
		&& 1 & 0 \\
		&&& 1
	\end{pmatrix} \ \tsupgp\  G''=1=[G',G'] \]
	
	Derived series: length 2\\
	$G$ is 2-step solvable.\\
	\[ [G,G'] \subset \left<E_{4,1} = \begin{pmatrix}
		1 & 0 & 0 & 1\\
		& 1 & 0 & 0 \\
		&& 1 & 0 \\
		0 &&& 1
	\end{pmatrix}\right> \]
	
	Central series:\[
	\begin{matrix}
		\begin{pmatrix}
		1 & \star & \star & \star\\
		& 1 & \star & \star\\
		&& 1 & \star \\
		&&& 1
		\end{pmatrix}&
		\tsupgp &
		\begin{pmatrix}
		1 & 0 & \star & \star\\
		& 1 & 0 & \star\\
		&& 1 & 0 \\
		&&& 1
		\end{pmatrix}& \tsupgp &
		\begin{pmatrix}
		1 & 0 & 0 & \star\\
		& 1 & 0 & 0\\
		&& 1 & 0 \\
		&&& 1
		\end{pmatrix}& 
		\tsupgp & 1\\
		&& [G,G] && [G,[G,G]]
	\end{matrix}
	\]
	
	\textbf{Computations}
	\begin{align*}
		[a,f] & = 1 & <b,c,e>\  \mathrm{abelian}\\
		[a,d] & = b & c\ \mathrm{central}\\
		[d,f] & = e\\
		[a,b] &= c\\
		[a,e] &= c
	\end{align*}
	\begin{align*}
		G \cong < a,d,f,c,b,e \mid & [a,f], c=[a,b]=[a,e],\\&e=[d,f],b=[a,d],
		\\&[b,c],[b,e],[c,e],[c,a],[c,d],[c,f]
	\end{align*}
	
	If $G$ is a finitely generated abelian group then there are $r \in \N_0$ and $(e_{i,j})_{i,j \in \N}$ almost all $0$.\\
	$(p_1=$1st prime, $p_2=$2nd$, \ddd)$.\\
	\[G=\Z^r \times \sum_{i,j \in \N_0} \left( \Z_{p_i^j} \right)^{e_{i,j}} \] amd $r,e_{i,j}$ are unique.
	
	\section*{Lecture 2019.12.09}
	\textbf{The Symmetric Group}
	\[S_n=(\{1 \ddd n\})=\{f \in \{1 \ddd n\}^{1 \ddd n} \mid f\ \mathrm{bij} \}\]
	$S_n \tsupgp A_n \to 1$\\
	$S_n \to \mathrm{\ sgn\ } \Z_2 \tsupgp 1$\\
	sgn$(\pi)=(-1)^{\mathrm{\#\ transpositions\ in\ } \pi}\\
	=\det p(\pi)\\
	\begin{aligned}
		p:S_n& \to \mathrm{GL}(n,k)\\
		& chak \neq 2
	\end{aligned}$\\
	$\Z_2=\{0,1\}=\{-1,1\}$\\
	$[ S_n , S_n ] =A_n$
	
	\textbf{Missing a load of stuff}
	
	\begin{theorem}
		A simple group $F$F of order 60 is isomorphic to $A_5$.
	\end{theorem}
	\begin{proof}
		Let $\abs{G}=60$. Cases:
		\begin{enumerate}
			\item case $\exists U <_{\neq} G: \abs{G:U} \leq 5$
			\item case compl.
		\end{enumerate}
		Check these:
		\begin{enumerate}
			\item $G$ acts on $\quotient{G}{U},\abs{\quotient{G}{U}}=2,3,4,5=q$.
			\begin{align*}
				content...
			\end{align*}
		\end{enumerate}
	\end{proof}
	
	\section*{Tutorial 2019.12.10}
	\textbf{15.}
	\[\mathrm{GL} (k^n) \qquad \abs{k}=k\]
	\begin{align*}
		\abs{\mathrm{GL}(k^n)}&=(k^n-1)(k^n-k)(k^n-k^2)\ddd(k^n-k^{n-1})\\
		U(k^n)&=\left\{
		\begin{pmatrix}
			1 & \ddd & \star\\
			& \diagdown\\
			0 & \ddd & 1
		\end{pmatrix} \in \mathrm{GL}(k^n)\right\}\\
		\abs{U(k^n)}&=k.k^2.k^3 \ddd k^{n-1}=k^{n \choose 2}
	\end{align*}
	\textbf{16.}
	Amalgamated product. \[\underbrace{\Z_2}_u \times \underbrace{\Z}_a \times \underbrace{\Z}_b \]
	\[\Z=<u> \qquad u=<w>\]
	\[\Z \times \Z= <a,b \mid [a,b]>\]
	\textbf{17.}
	\[\begin{matrix}
		S(2,5)&=&\left\{\begin{pmatrix}
			a & c\\
			0 & b
		\end{pmatrix} \mid a,b,c \in \Z_5 \right\}\\
		\triangledown\\
		S(2,5)'&=&[S(2,5),S(2,5)]\\
		&=&N(5,2)\\
		&=&\left\{\begin{pmatrix}
			1 & c\\
			0 & 1
		\end{pmatrix} \mid c \in \Z_5\right\} & \cong \Z_5\\
		\triangledown\\
		1 & & G'=[G,G]=<[a,b]\mid a,b \in G>_G
	\end{matrix}\]
	\textbf{Further example:}
	Find the amalgamated product:
	\begin{align*}
		H_1 &= S_3=<a,b \mid a^3,u^2, uau=a^{-1}>\\
		H_2 &= \Z_9=<u \mid u^9>\\
		U&= \Z\times \Z = <v,w \mid [v,w]>\\
	\end{align*}
	\[\begin{matrix}
		U \to H_1 & U \to H_2\\
		v \mapsto (123)_{=a} & v \mapsto u\\
		w \mapsto 0 & w \mapsto 1
	\end{matrix}\]
	\begin{align*}
		&<a,b,u,v,w \mid a^3, b^2, baba, u^9, \underbrace{v=a, v=u}_{a=u},w=1>\\
		&\cong <a,b,\cancel{u} \mid a^3,b^2,baba,\cancel{u^9},\cancel{a=u}>\\
		&\cong H_1
	\end{align*}\\
	\\
	
	\section*{Lecture 2019.12.13}
	\begin{theorem}[A simple group of order 60 is $\cong A_5$]
	\end{theorem}
	\begin{proof}
		$\abs{G}=60$, $G$ not simple.
		\begin{enumerate}
			\item $U <_{\neq} G, \abs{G:U} \leq 5$\\
			transitive action of $G$ on $\quotient{G}{U}, \abs{\quotient{G}{U}}=2345\\
			G \to S_k\quad k=2,3,4 \quad \abs{G}=60,\ \abs{S_k}\leq 24$ not injective, $\overbrace{\ker}^{\neq =1} \tsubgp_{\neq} G$\\
			If $G \tsubgp S_5$ then $G=A_5$.
			$G \to S_5 \tsupgp A_5,\ G \to A_5$ injective (otherwise $\overbrace{\ker}^{\neq 1} \tsubgp\ G )$
			\item ALl subgroups $U < G$ have index $\geq 6$.\\
			$60=2^2.3.5 \qquad $2-Sylow subgroups $P$ have $\abs{P}=4$ and are Abelian.\\
			$(n_2=1 \mod 2, n_2 \mid 3.5, n_2=1,3,5,15)$\\
			$\begin{matrix}
				(0) & n_2 = 1 & p \tsubgp G & \implies G\ \mathrm{not\ simple}\\
				(2) & n_2 > 1 & p,Q \in \mathrm{Syl}_2(G)
			\end{matrix}$\\
			$1\neq y \in P \cap Q \cong \Z_2,\ Q=xPx^{-1},\quad P,Q$ Abelian.
			\[\begin{matrix}
				& &     4 & &         3.4,\ 5.4 & & & & < 6 & 60\\
				1 & < & p & <_{\neq} & <P,Q> & = & U & & < & G\\
				& & y \in & & y \in 
			\end{matrix}\]
			$\implies \exists U < G$, index $U < 6$.
			\skipitems{-2}
			\item If $P,Q \in \mathrm{Syl}_2(G) \implies P \cap Q=1\\
			\begin{matrix}
				& 4 & 15 \to \\  
				1 < & P & \tsubgp & N_GP & < & G\\
				& & 1,3,5,15 
			\end{matrix}$
			\begin{align*}
				\abs{N_GP:P}&=15 \implies N_GP=G,\ P\tsubgp G\\
				&=3,5 \implies \abs{N_GP}=12.20 \implies \abs{G:N_GP}=5,3 < 6\\
				&=1 N_G(P)=P
			\end{align*}
			$G \circlearrowleft \underbrace{\mathrm{Syl}_2}_\mathrm{some\ orbit}G$ transitive\\
			\[\quotient{\abs{G}}{\abs{\mathrm{Stabiliser}}}=\abs{\mathrm{orbit}}\]
			\[\frac{\abs{G}}{\abs{N_GP}} =n_2,\qquad \frac{60}{4} =n_2 = 15 \]
			\[\#\{g \in G \mid \abs{g}=2,4\} = \#\{g \in G \mid 1 \neq g\ \mathrm{in\ a\ 2-Sylow\ subgroup}\}\]
			\[=n_2.(\abs{P}-1)=15\times 3=45\]
			$G$ has 45 elements of order 2 or 4.
			$\mathbf{p=5}:\ n_5=1\mod 5=1,6,13\ddd\\
			n_5 \mid 12\\
			\implies n_5 = 1$ or $6$.\\
			$n_5 = 1 \implies P_5 \tsubgp G$, $G$ not simple.\\
			$n_5=6\quad P_5 \cong \Z_5$,\\
			If $P_5,P''_5 \in \mathrm{Syl}_5G\\
			$ then $P_5=P'_5$ or $P_5 \cap P'_5=1\\
			\#\{ g \in G \mid \abs{g}=5 \}  = 4 \times n_5 = 24 \\
			24+45+1 > 60$
		\end{enumerate}
	\end{proof}

	\begin{theorem}
		$G$ finite group, $\abs{G}=p_1^{e_1} \ddd p_k^{e_k},\quad p_1<p_2< \ddd < p_k$ prime.\\
		$P < G\ p$-Sylow subgroup if $\abs{p}=p^r,\ G=p^rm,\ p\nmid m$.\\
		$\mathrm{Syl}_p(G)=\{p\ \mathrm{Syl.\ subgroups}\}$\\
		$n_p(G)=\#\mathrm{Syl}_p(G)$.
		\begin{theorem}[Sylow]
			$n_p \equiv 1 \mod p,\ n_p \mid m\\
			G \circlearrowleft\ \mathrm{Syl}_p(G)$ by conj. transitively\\
			every $p$ subgroup of $G$ is subgroup of a $p$-Sylow subgroup.
			\begin{corollary}
				$p$-Sylow subgroups are \textbf{maximal} $p$-subgroups w.r.t. $c$.
			\end{corollary}
		\end{theorem}
	\end{theorem}
	\begin{definition}[Nilpotent]
		\[\begin{matrix}
			G & \to & \quotient{G}{Z(G)} & \to & \quotient{\quotient{G}{Z(G)}}{Z(\quotient{G}{Z(G)})} & \to \ddd & 1\\
			G & > & G'=[G,G] &\tsupgp & [G,[G,G]] & \tsupgp & [G,[G,[G,G]]] & \tsupgp \ddd \tsupgp & 1
		\end{matrix}\]
		"iterated central extensions", lower central series.
	\end{definition}
	Nilpotent groups are products of $p$-groups.
	
	\begin{theorem}
		$G$ a finite group. Tfae
		\begin{enumerate}
			\item $G$ nilpotent
			\item $H <_{\neq } G \implies G <_{\neq} N_G(H)$
			\item $G \cong  \Pi_{p\ \mathrm{a\ sylow\ subgroup\ of\ } G} (p)\quad (P \tsubgp G)\\
			\cong \Pi_{q\ \mathrm{prime},\ q \mid \abs{G}}(p_q)$
		\end{enumerate}
	\end{theorem}
	\begin{proof}
		$1 \to 2$, $G$ nilpotent, $H < G$.\\
		If $Z(G \nleq H, H<_{\neq}Z(G).H < N_GH\\
		G_0 \tsupgp G_1 \tsupgp . \ddd \ddd Z(G) \tsupgp 1\\
		H \tsupgp G \cap G_1 \tsupgp \ddd G \cap Z(G) \tsupgp 1$\\
		If $Z(G) < H$, then $\quotient{H}{Z(G)} < \quotient{G}{Z(G)}$\\
		By induction on the leny that the central series ("nilpotency class")
		\[ \quotient{H}{Z(G)} <_{\neq} N_{\quotient{G}{Z(G)}}\left(\quotient{H}{Z(G)}\right) \implies H <_{\neq} N_G(H) \]
		\textbf{Frattini argument}\\
		$P < N < G, p \in \mathrm{Syl}_p(N) \circlearrowleft N, N \mathrm{conj} N, N \circlearrowleft G\  \mathrm{conj}$\\
		$\implies G=N.N_G(p)$.\\\\
		$G=N_G(N_G(p)))\\
		G > N_G(p) > p$\\
		$2 \to 3,\ P<G\ p-$Sylow subgroup.\\
		By (2), $N_G(N_G(p)) >_{neq} N_G(p)$ if $N_G(p) \neq G$.\\
		By the Frattini argument:\\
		$N_G(p.N_G(p))=N_G(p)$.\\
		Hence $N_G(p)=G$.
	\end{proof}

	\begin{definition}
		?
	\end{definition}

	\section*{Lecture 2019.12.20}
	If $G$ is a group with $\abs{G}=300$ elements, then $G$ is not simple.
	\begin{proof}
		\textbf{Sylow's Theorem:} $\abs{G}=p^r.m,\ p \nmid m,\ p$ prime\\
		then $\mathrm{Syl}_p(G):=\{P<G \mid \abs{P} = p^r\} = \{P < G \mid P\ \mathrm{a}\ p\mathrm{-group}, p\ \mathrm{maximal} \}\\
		n_p(G)=\#\mathrm{Syl}_p(G)$
		\begin{enumerate}
			\item $n_p \equiv 1 \mod p,\ n_p \mid m$
			\item $Q < G, \abs{Q}=p^l,\ \mathrm{some}\ l$ then $Q < P$ for some $p \in \mathrm{Syl}_p(G)$
			\item $G$ transitive on $\mathrm{Syl}_p(G)$ (via conjugation).\\
			$P,\tilde{P} \in \mathrm{Syl}_p(G) \implies \exists g \in G : \tilde{P}=gPg^{-1}$
		\end{enumerate}
	
		So back to proof:\\
		$400=4\cdot3\cdot25=2^2\cdot3\cdot5^2\\
		n_2 \equiv 1 \mod 2,\ n_2 \mid 3 \cdot
		1,3,5,\cancel7, \cancel9, \cancel{11},\ddd,15,25,75\\
		n_5 = 1 \mod 5,\ n_5 \mid 12 \implies n_5 = 1, \cancel2, \cancel3, \cancel 4, 6, \cancel{12}\\
		$a)$n_5 = 1 \to $ unique Syl $\implies$ normal, $G$ not simple\\
		b)$n_5=6\to G \circlearrowleft\mathrm{Syl}_5(G) \cong \{1,2,3,4,5,6\},\ G \to^\phi S_6$.
		
		\textbf{Orbit-Stabiliser Theorem}
		$G \circlearrowleft X \ni x_0\\
		\underbrace{G.x_0}_{\mathrm{orbit}} \times \underbrace{G_{x_0}}_{\mathrm{stabiliser}} \sim G\\
		\#G.x_0 \cdot \# G_{x_0} = \# G\\\\
		\begin{aligned}
			\quotient{G}{G_{x_0}} &\to G.{x_0}\\
			gG_{x_0} & \mapsto g.x_0
		\end{aligned}\\\\
		$ If $\phi$ not injective, then $\underbrace{\ker \phi }_{\neq 1}\tsubgp G$, hence $G$ not simple.\\
		If $\phi$ injective, then $\underbrace{G}_{300} {\to^{\cong}} \underbrace{\phi(G)}_{300} < \underbrace{S_6}_6!$
		
		\textbf{Homoemorphism Theorem}
		\[ 1 \to \ker \phi \to{\tsubgp} G \to^\phi H \qquad \mathrm{gp. hom.} \]
		\begin{align*}
			\quotient{G}{\ker \phi} & \cong \phi(G)\\
			g \ker \phi &\mapsto \phi(g)
		\end{align*}
		
		$H<G,\ G \circlearrowleft \quotient{G}{H}=X\\
		H=G_H\\
		\quotient{G}{H} \times H \sim G$ orbit.
		
		$1 < \underbrace{P}_\mathrm{small} \tsubgp \underbrace{N_GP}_\mathrm{big} < \underbrace{G}_{\begin{matrix}
			\circlearrowleft\\
			\mathrm{Syl}_p = \quotient{G}{N_GP}
			\end{matrix}} \to S_{n_p}=S)2,S)3,S_4,S_5 \tsupgp A_5$.
		
		If $P$ is a $p$-Sylow subgroup of $G$, $P \tsubgp G$ then $P$ is characteristic.
	\end{proof}
	
	
	
\end{document}