\documentclass{article}

\usepackage{amsmath}
\usepackage{amssymb}
\usepackage{amsthm}

\title{MT441P Groups II - Assignment 1}
\author{Ciarán Ó hAoláín - 17309103 - ciaran.ohaolain.2018@mumail.ie}

\let\nset\varnothing
\let\ddd\cdots

\newcommand{\sth}{\mathrm{s.th\ }}
\newcommand{\R}{\mathbb{R}}
\newcommand{\N}{\mathbb{N}}
\newcommand{\Z}{\mathbb{Z}}
\newcommand{\C}{\mathbb{C}}
\renewcommand{\H}{\mathbb{H}}
\newcommand{\quotient}[2]{{\raisebox{.2em}{$#1$}\left/\raisebox{-.2em}{$#2$}\right.}}
\newcommand{\abs}[1]{\left|#1\right|}

\newcommand{\stab}{\mathrm{stab}}
\newcommand{\orb}{\mathrm{orb}}
\newcommand{\gp}{\mathrm{gp}}
\newcommand{\subgp}{\ \mathrm{subgp.}}

\newtheorem{theorem}{Theorem}[section]
\newtheorem{corollary}{Corollary}[theorem]
\newtheorem{lemma}[theorem]{Lemma}
\theoremstyle{definition}
\newtheorem{definition}{Definition}[section]
\theoremstyle{remark}
\newtheorem*{remark}{Remark}
\theoremstyle{example}
\newtheorem*{example}{Example}

\begin{document}
	\maketitle
	\begin{enumerate}
		\item 
		\begin{enumerate}
			\item 
			\[x=\begin{pmatrix}
				1 & 1 & 0\\
				0 & 1 & 0\\
				0 & 0 & 1
			\end{pmatrix}, \quad y=\begin{pmatrix}
				1 & 0 & 1\\
				0 & 1 & 0\\
				0 & 0 & 1
			\end{pmatrix}, \quad z=\begin{pmatrix}
				1 & 0 & 0\\
				0 & 1 & 1\\
				0 & 0 & 1
			\end{pmatrix}\]
			\[ G_a=\left< x,y,z | xyx^{-1}y^{-1},yzy^{-1}z^{-1}\right> \]
			\item While the vectors in the group span $\{(a,b,c)|a,b,c \in \Z\}$, we only need two generators since if $x=(1,0,0)$ and $z=(0,0,1)$, we have that $zxz=(0,1,0)$.
			\[ G_b=\left< x,z | z^2,xzxzx^{-1}z^{-1}x^{-1}z^{-1} \right> \]
			\item $G_C \cong A_4$
			\[ G_c=\left<a,b|a^3,b^2,(ab)^3\right> \]
			\item With unique prime factorisations, we can write the entire set of rational numbers using those and their inverses (the relation $x.\frac{1}{x}=1$ is trivial). Let $P$ be the set of all prime numbers.\\
			This gives \[ G_d = \left<P, \varnothing \right> \]
		\end{enumerate}
		\item No, this group can not be finite. There's not enough relations here to confine the number of words generated by the three generators to a finite number. There are an infinite number of words that can be created using three generators and $\#R<=2$ .
		\item 
	\end{enumerate}
\end{document}